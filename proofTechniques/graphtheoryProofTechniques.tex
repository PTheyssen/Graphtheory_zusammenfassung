\documentclass[a4paper]{article}
\usepackage[english]{babel}
\usepackage[utf8]{inputenc}
\usepackage[T1]{fontenc}
\usepackage[autostyle=true,german=quotes]{csquotes}
\usepackage{amsmath}
\usepackage{amssymb}
\usepackage{amsthm}
\usepackage{oldgerm}

\renewcommand{\labelenumi}{(\roman{enumi})}
\title{Graphtheory Proof Techniques}
\author{Klaus Philipp Theyssen}

\begin{document}
\begin{center}
    \Large Graphtheory Proof Techniques
\end{center}


\section{General Problem Solving, Testprep}
\begin{enumerate}
    \item Look at small examples 
    \item what similar problems theorems do I know what is different
    \item take a step back and dont get fixiated on the first idea and try 
        a completly different way of looking at the problem
    \item start with the most difficult problems
    \item work on a problem as long as I can make progress, if I do not know
        how to continue switch to another problem
    \item see solution in big picture, \enquote{does this make sense?}
    \item relax while taking test, focus on breathing 
    \item see test as sports event to show of all the hard work in the past
\end{enumerate}

\section{Basic Combinatorics}
\paragraph{Fundamental Principle of Counting (Multiplicationrule)} $ $ \\
We want to construct $k$-Tupel $ (a_1,a_2,...,a_k) $ where the $ k $ places of the 
tuple are filled left to right. With $ j_1 $ being the amount of choices for $ a_1 $
and $ j_i $ the amount of choices of $ a_i $ with $ i \in \{1,...,k\} $

$$ j_1 \cdot j_2 \cdot ... \cdot j_k \text{ different } k\text{-Tupel} $$

\paragraph{Example: Lotto} $ $ \\  
6 Numbers out of 49 are being drawn in succesion. \\
Therefore $ 49 \cdot 48 \cdot 47 \cdot 46 \cdot 45 \cdot 44
 = 10\text{ }068 \text{ }347 \text{ }520 $ different Tuples.

\paragraph{$k$-Permutation with order}
$$ |M|^k = |\{(a_1,a_2,...,a_k): a_j \in M \text{ for } j = 1,...,k\}| $$

\paragraph{$k$-Permutation without order}
$$ \{(a_1,...,a_k): a_i \neq a_j \text{ for } 1 \leq i \neq j \leq k\} $$

\section{Double Counting}
Describing a finite set $ X $ from two perspectives, with two Formulas.
Since the two Formulas/Expressions equal the size of the same set they equal each other.

\smallskip
If the sets are different, you can try to show the sets are bijective.
Then the bijection rule shows that the expressions are equal.

\paragraph{Handshake Lemma}
$$ \sum_v d(v) = 2e $$ 

\begin{proof}
    We can count the size of the set of vertex-edge incidences in two ways.
    By summing the degrees of the vertices or by counting two incidences for every edge.
    Since we are counting the same set these expression must be equal (double counting).
\end{proof}

\smallskip
From this follows that an undirected graph contains an even number of 
vertices of odd degree, since an odd number would mean \\ $ (2e) \text{ mod } 2 = 1 $ 
\textbf{contradiction}. 

\paragraph{Counting trees}
What is the number $ T_n $ of different trees that can be formed with n vertices,
counting also isomorphic trees (naming of vertices is important)?

$$ \text{Cayley's formula: } T_n = n^{n - 2} $$ 
\begin{proof}
Count the number of different sequences of directed edges that can be added to an empty
graph on n vertices to get an rooted tree.


\end{proof}


\section{Minimality/Maximality of Object}
Assume that you have a shortest or longest walk,path and show that you can 
get a contradiction by a constructing a shorter / longer path.

\section{Induction}
\paragraph{forward Induction}

\paragraph{multiple basescases}


\section{List of important Graphs}


\end{document}
