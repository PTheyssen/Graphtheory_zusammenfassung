\section{problems}

\paragraph{problem 1}
Determine the number of edges, average degree, diameter and girth of the 
\textit{d-dimensional hypercube}.
\begin{proof}
    For number of edges consider vertex degree, for average degree observe that 
    $ Q_d $ is regular.

    \bigskip
    \noindent claim: $ Q_d $ has diameter $ d $.

    \smallskip
    \noindent First prove that for any $x,y \in V$ the distance $d(x,y)$ in 
    $Q_d$ is the number of postions where they differ.
    Suppose they differ in $l \geq 1$ 
    positions then we get a path of length $ l $ by inverting each entry where
    they differ sequentially. There can not be a shorter Path since each step 
    can only change one position. \\
    Therefore the diameter is $d$, because thats the maximum amount of positions 
    where two sequences can differ, consider $ 00...0 $ and $ 11...1 $.

    \bigskip
    \noindent claim: girth of $Q_d$ is $ \infty $ if $ d = 1 $ and otherwise 4.

    \smallskip
    \noindent
    If $ d = 1 $, the resulting graph $Q_1$ is acyclic. If $d \geq 2$ we first 
    observe that then $ Q_d $ is triangel-free. Suppose otherwise than there is 
    a triangle $xyz$ in $ Q_d $, now consider the amount of 1's in $|v|,v\in V$.
    Wlog $ |x| $ is even then $ |y| $ must be odd and $ |z| $ must be even.
    But $ zx $ is an edge, a contradiction. \\
    On the other hand $ Q_d $ does contain 4-cycles:
    $$ (00...0, 010...0, 100...0, 00...0)$$
\end{proof}

\paragraph{problem 2} 
Show that any tree $ T $ has at least $ \Delta(T) $ leaves. 
\begin{proof}
Induction on $ n = |T| $.
\\
\textbf{Base:} tree on two vertices has 2 leaves and maximum degree is 1.

\smallskip \noindent 
\textbf{Step:} remove a leaf $v$ from $T$ and let $u$ be its neighbour,
so that $ T^\prime := T - v $ we than have two cases:

\smallskip \noindent
case 1: $ \Delta(T) = \Delta(T^\prime) $ \\
We know by I.H. that $ T^\prime $ has at least $ \Delta(T^\prime) $ leaves, thus 
if the maximum degree of $ T $ is the same the claim holds.

\bigskip \noindent
case 2: $ \Delta(T) = \Delta(T^\prime) + 1 $
This means $ u $ is the vertex with maximal degree in $T$, but we know that the
leafs of $T$ can only be the leafs of $T'$ plus the leaf $ v $ we removed
therefore $ T $ has $ \Delta(T^\prime) + 1 $ leafs.

\end{proof}
\begin{proof} By Counting $ $ \\
Let $ T $ be any tree. Let $ L \subseteq V $ be the set of leaves and 
$ N = V \setminus L $ the set of non-leaves in $ T $. Let $ u $ be a vertex 
with maximum degree $ \Delta(T) $. We know that a tree has $ |V| + 1 $ edges.

\begin{align}
    2 \cdot (|V| - 1) &= \sum_{v \in V(T)} d(v) \\
    &= d(u) + \sum_{v \in L} d(v) + \sum_{v \in N \setminus \{u\}} d(v) \\
    &\geq \Delta(T) + \sum_{v \in L} 1 + \sum_{v \in N \setminus \{u\}} 2 \\
    &= \Delta(T) + |L| \cdot 1 + (|V| - |L| - 1) \cdot 2  \\
    &= \Delta(T) + 2 \cdot (|V| - 1) - |L|
\end{align}
From this follows $ |L| \geq \Delta(T) $, as desired.
\end{proof}

\paragraph{problem 3}
Prove that either a graph or its complement is connected.
\begin{proof}
    Let $ G = (V,E) $ be any non-empty graph. We assume $G$ is not connected and 
    shall argue that $ \overline{G} $ is connected. \\
    Since $G$ is disconnect, we find two vertices $ u,w \in V $ and a connected
    component $ C $ of $ G $ such that $ u \in C $ and $ w \notin C $. Now in 
    $ \overline{G} $ all vertices in $ C $ are adjacent to $ w $. And in 
    particular $uw\in E(\overline{G})$, so all vertices lie in a single 
    connected component of $ \overline{G} $, which is therefore connected.
\end{proof}


\paragraph{problem 4}
Prove that the vertex set of any graph can be partitioned into two sets such 
that for each vertex, at least half of its neighbors belong to the other set.
\begin{proof}
Consider partition of $ V(G) $ into disjoint sets $ A,B $, maximizing the number 
of edges between $A$ and $B$. We will show that moving any vertex to the other 
sets results in a partition that has less edges 
between the sets than the original one.

\smallskip \noindent
Pick wlog a vertex $ v \in A $. Let $ d_B = N(v) \cap B $ and $ d_A = d - d_B $.
Now consider $ A^\prime = A - v $ and $ B^\prime = B + v $
$$ |\{uw \in E: u \in A^\prime, w \in B^\prime\}| =
|\{uw \in E: u \in A, w \in B\}| - d_B + (d - d_B) $$
but by the maximality of edges between $ A $ and $ B $ we have:
$$|\{uw \in E: u \in A^\prime, w \in B^\prime\}| \leq 
|\{uw \in E: u \in A, w \in B\}|$$
which gives us 
$$ -d_B + (d - d_B) \leq 0 \iff d_B \geq d/2 $$ 
\end{proof}


\paragraph{problem 5}
Show that the following statements are equivalent
\begin{enumerate}
    \item $ G $ is connected, but $ G - e $ is disconnected for every edge $ e $
    \item Any two vertices in $ G $ are linked by a unique path
\end{enumerate}
\begin{proof} $ $ \\
    \enquote{(ii) $\Rightarrow$ (i)}: trivial

    \smallskip \noindent
    \enquote{(i) $\Rightarrow$ (ii)}: $ $ \\
    As $G$ is connected, there is at least one path between any two vertices
    of $G$. Lets assume for the sake of contradiction that there
    are some vertices $ x $ and $y$ that are joind by at least two
    paths $ P_1, P_2$. As $ P_1 \neq P_2 $ there 
    is and edge $e_0$ that lies in $P_1$ but not in $ P_2$. We shall show that 
    the graph $ G - e_0 $ is still connected, which will be a contradiction.

    \smallskip \noindent
    Idea is to consider the path between any two vertices $u,v$ and remove 
    the edge $ e_0 $ and still show we can reach $ v $ from $ v $ by using 
    the vertices $ x $ and $ y $ which have two paths between them and simply put 
    $ e_0 $ in one of them, then we can user the other one by going from one 
    endpoint  from $e_0$ to $x$, possible because 
    $ G $ is connected. Then to $y$ using the other path and then from 
    $y$ to the other endpoint of $e_0$ and finally to $v$.
    Thus $u$ and $v$ are still connected after removing $e_0$ a contradiction. 

\end{proof}

\paragraph{problem 6}
Let $ T_1,...,T_k $ be subtrees of a tree $ T $, any two of which have at least 
one vertex in common. Prove that there is a vertex common to all $ T_i$.
\begin{proof}
   Apply induction on $|T|$. If $|T| = 1$ then $T$ consists of a single vertex,
   say $ v$. If $ T_1,...,T_k $ are subtrees of $ T $ with pairwise intersecting
   vertex sets, then we must have $T_i = T$ for each $ i = 1,...,k$. It follows 
   that $ v $ belongs to each $ T_i$.
   
   \smallskip \noindent
   So assume $ n \geq $ is an integer, and let us suppose this result holds for 
   all trees of order $ n - 1 $. Suppose $ T $ is a tree with $ |T| = n $ and 
   $ T_i,...,T_k $ are subtrees of $ T $. We assume $ k \geq 2 $. Let $ v $ be a 
   leaf in $ T $ and let $ T' = T - v $ be the tree resulting from removing
   this leaf. 
\end{proof}

\paragraph{Problem 15}
Let $G$ be a nonempty connected graph and let $x \in V(G)$. For each $r \geq 0$
let $B_r = B_r(x)$ be the set of vertices in $G$ distance exactly $r$ from 
$x$. Prove that for some $r$:
$$ \chi(G) \leq \chi(G[B_r]) + \chi(G[B_{r+1}]) $$
\begin{proof} $ $\\
Proof sketch: color layer by layer use induction create proper coloring from 
from one with one less layer. Take new colors from $\{1,...,\chi(B_r)
+ \chi(B_{r+1})\}$ these exists otherwise choice of $r$ not maximal. 

\bigskip
Proof: \\
Let $x \in V(G)$ and note that since $G$ is connected, for som $t \geq 0$ the 
sets $B_0 = \{x\},...,B_t$ partition the vertex set. By the definition of 
each $B_r$, there are no edges in $G$ between $B_i$ and $B_j$ whenever 
$|i-j| > 1$. In other words, the only edges in $G$ are possibly inside each 
$B_i$ and between consecutive $B_i's$. For simplicity of notation we write 
$B_r$ for $G[B_r]$. Now let $r \geq 0$ be such that $\chi(B_r) + \chi(B_{r+1})$
is maximized and write 
$$ C = \{1,...,\chi(B_r) + \chi(B_{r+1})\} $$
Our aim is to color $G$ 'layer by layer'. Assign an arbitrary color from $C$
to $B_0 = \{x\}$. Now assume that $B_0,...,B_{k-1}$ have been colored for some
$k \geq 1$ with $\chi(B_0),...,\chi(B_{k-1})$ colors from $C$, respectiviely,
such that the graph $G[B_0 \cup ... \cup B_{k-1}]$ is properly colored. Let us 
show how to color $B_k$ with $\chi(B_k)$ colors from $C$ such that 
$G[B_0 \cup ... \cup B_k]$ is properly colored. By assumption, there is some
subset $C' \subseteq C$ of $\chi(B_{k-1})$ colors used to color $B_{k-1}$ in 
the coloring constructed so far. We claim that $\chi(B_k)\leq |C\setminus C'|$.
Indeed, if otherwise then
$$ \chi(B_{k-1} + \chi(B_k) > |C'| + (|C| - |C'|) = |C| = \chi(B_r)
+ \chi(B_{r+1}) $$
contradicting the maximal choice of $r$. Therefore we max choose a set of 
$\chi(B_k)$ colors from $C \setminus C'$. These colors do not appear in the 
coloring of $B_{k-1}$, so this is a legal coloring of the graph induced on
$B_{k-1} \cup B_k$. As we observed earlier, there are no edges between 
non-consecutive $B_i$'s so this defines a proper coloring of $G$ with 
$|C| = \chi(B_r) + \chi(B_{r+1})$ colors.
\end{proof}

\paragraph{Problem 16}
Let $G$ be a connected graph with minimum degree $\delta(G) = k \geq 1$. Prove
that $G$ contains a path $x_1,...,x_k$ such that $G - \{x_1,...,x_k\}$ is 
also connected. 

\bigskip
Proof sketch: consider longest path, suppose removing path of length $k$ is 
disconnecting $G$, consider the two components. Consider a new longest path 
in one of the components starting at a vertex $u$ with maximum number of nbhs,
and we use that to build a longer path in $G$,
from this we get a inequality which has to hold to not get contradiction,
but if the inequality holds, then get contradiction of choice of $u$.

\begin{proof}
    Consider a longest path $P = x_1,..,x_l$. As every neighbor of $x_l$ lies 
    on $P$, we must have $l \geq k+1$ Suppose,by way of contradiction, that 
    $G' = G - \{x_1,...,x_k\}$ is disconnected. Therefore, $G'$ has at least 
    two components; let $C$ be the component not containing $x_{k+1}$
    Let $u \in C$ have the maximum number, say $d$, of neighbors in 
    $\{x_1,...,x_k\}$. Note that $u$ cannot be joined to $x_1$ as otherwise we 
    obtain a path longer than $P$. Also $u$ has at least one neighbor in 
    $\{x_1,...,x_k\}$ (otherwise $G$ is disconnected).
    It follows that $\delta(G[C]) \geq k-d > 0$ since vertices in $C$ only have
    neighbors in $\{x_1,...,x_k\} \cup C$ and $\delta(G) = k$  and 
    $\delta(G) = k$. Let $x_i$ be a neighbor of $u$ for some $i \in [k]$ and 
    consider a longest path $Q$ in $G[C]$ that starts with $u$. This path 
    has length at least $k-d$. Indeed, the last vertex of $Q$ has all its 
    neighbors in the path $Q$, and $\delta(G[Q]) \geq k-d$. Lets try to finde
    a longer path in $G$ using $Q$. If $u'$ denotes the last vertex of $Q$, 
    consider the path $u'Qux_i...x_l$. It has length at least 
    $$ (k-d) + 1 + (l-i) $$
    As $P$ was a longest path in $G$, this is a contradiction unless 
    $$ (k-d) + 1 + (l-i) \leq l-1 $$
    which is equivalent to $i \geq k - d+2$. But then the neighborhood of $u$
    in $\{x_1,...,x_k\}$ is contained in $\{x_{k-d+2},...,x_k\}$, which has 
    size $k-(k-d+2)+1 = d-1 < d$, contradictin our choice of $u$. Thus
    $G-\{x_1,...,x_k\}$ must be connected.
\end{proof}
