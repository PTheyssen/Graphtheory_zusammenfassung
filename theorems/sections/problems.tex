\section{problems}

\paragraph{problem 1}
Determine the number of edges, average degree, diameter and girth of the 
\textit{d-dimensional hypercube}.
\begin{proof}
    For number of edges consider vertex degree, for average degree observe that $ Q_d $ 
    is regular.

    \bigskip
    \noindent claim: $ Q_d $ has diameter $ d $.

    \smallskip
    \noindent First prove that for any $ x,y \in V $ the distance $ d(x,y) $ in $ Q_d $ is 
    the number of postions where they differ. Suppose they differ in $ l \geq 1 $ 
    positions then we get a path of length $ l $ by inverting each entry where they 
    differ sequentially. There can not be a shorter Path since each step 
    can only change one position. \\
    Therefore the diameter is $ d $, because thats the maximum amount of positions 
    where two sequences can differ, consider $ 00...0 $ and $ 11...1 $.

    \bigskip
    \noindent claim: girth of $ Q_d $ is $ \infty $ if $ d = 1 $ and otherwise 4.

    \smallskip
    \noindent
    If $ d = 1 $, the resulting graph $ Q_1 $ is acyclic. If $ d \geq 2 $ we first 
    observe that then $ Q_d $ is triangel-free. Suppose otherwise than there is 
    a triangle $ xyz $ in $ Q_d $, now consider the amount of 1's in $ |v|, v \in V $.
    Wlog $ |x| $ is even then $ |y| $ must be odd and $ |z| $ must be even.
    But $ zx $ is an edge, a contradiction. \\
    On the other hand $ Q_d $ does contain 4-cycles:
    $$ (00...0, 010...0, 100...0, 00...0)$$
\end{proof}

\paragraph{problem 2} 
Show that any tree $ T $ has at least $ \Delta(T) $ leaves. 
\begin{proof}
Induction on $ n = |T| $.
\\
\textbf{Base:} tree on two vertices has 2 leaves and maximum degree is 1.

\smallskip
\noindent \textbf{Step:} remove a leaf $ v $ from $ T $ and let $ u $ be its neighbour,
so that $ T^\prime := T - v $ we than have two cases:

\smallskip \noindent
case 1: $ \Delta(T) = \Delta(T^\prime) $ \\
We know by I.H. that $ T^\prime $ has at least $ \Delta(T^\prime) $ leaves, thus 
if the maximum degree of $ T $ is the same the claim holds.

\bigskip \noindent
case 2: $ \Delta(T) = \Delta(T^\prime) + 1 $
This means $ u $ is the vertex with maximal degree in $ T $, but we know that the
leafs of $ T $ can only be the leafs of $ T^\prime $ plus the leaf $ v $ we removed
therefore $ T $ has $ \Delta(T^\prime) + 1 $ leafs.

\end{proof}
\begin{proof} By Counting $ $ \\
Let $ T $ be any tree. Let $ L \subseteq V $ be the set of leaves and 
$ N = V \setminus L $ the set of non-leaves in $ T $. Let $ u $ be a vertex 
with maximum degree $ \Delta(T) $. We know that a tree has $ |V| + 1 $ edges.

\begin{align}
    2 \cdot (|V| - 1) &= \sum_{v \in V(T)} d(v) \\
    &= d(u) + \sum_{v \in L} d(v) + \sum_{v \in N \setminus \{u\}} d(v) \\
    &\geq \Delta(T) + \sum_{v \in L} 1 + \sum_{v \in N \setminus \{u\}} 2 \\
    &= \Delta(T) + |L| \cdot 1 + (|V| - |L| - 1) \cdot 2  \\
    &= \Delta(T) + 2 \cdot (|V| - 1) - |L|
\end{align}
From this follows $ |L| \geq \Delta(T) $, as desired.
\end{proof}

\paragraph{problem 3}
Prove that either a graph or its complement is connected.
\begin{proof}
    Let $ G = (V,E) $ be any non-empty graph. We assume $ G $ is not connected and 
    shall argue that $ \overline{G} $ is connected. \\
    Since $ G $ is disconnect, we find two vertices $ u,w \in V $ and a connected
    component $ C $ of $ G $ such that $ u \in C $ and $ w \notin C $. Now in 
    $ \overline{G} $ all vertices in $ C $ are adjacent to $ w $. And in 
    particular $ uw \in E(\overline{G}) $, so all vertices lie in a single connected
    component of $ \overline{G} $, which is therefore connected.
\end{proof}


\paragraph{problem 4}
Prove that the vertex set of any graph can be partitioned into two sets such that for
each vertex, at least half of its neighbors belong to the other set.
\begin{proof}
Consider partition of $ V(G) $ into disjoint sets $ A,B $, maximizing the number 
of edges between $ A $ and $ B $. We will show that moving any vertex to the other sets 
results in a partition that has less edges between the sets than the original one.

\smallskip \noindent
Pick wlog a vertex $ v \in A $. Let $ d_B = N(v) \cap B $ and $ d_A = d - d_B $.
Now consider $ A^\prime = A - v $ and $ B^\prime = B + v $
$$ |\{uw \in E: u \in A^\prime, w \in B^\prime\}| =
|\{uw \in E: u \in A, w \in B\}| - d_B + (d - d_B) $$
but by the maximality of edges between $ A $ and $ B $ we have:
$$|\{uw \in E: u \in A^\prime, w \in B^\prime\}| \leq 
|\{uw \in E: u \in A, w \in B\}|$$
which gives us 
$$ -d_B + (d - d_B) \leq 0 \iff d_B \geq d/2 $$ 
\end{proof}


\paragraph{problem 5}
Show that the following statements are equivalent
\begin{enumerate}
    \item $ G $ is connected, but $ G - e $ is disconnected for every edge $ e $
    \item Any two vertices in $ G $ are linked by a unique path
\end{enumerate}
\begin{proof} $ $ \\
    \enquote{(ii) $\Rightarrow$ (i)}: trivial

    \smallskip \noindent
    \enquote{(i) $\Rightarrow$ (ii)}: $ $ \\
    As $ G $ is connected, there is at least one path between any two vertices of $ G$.
    Lets assume for the sake of contradiction that there are some vertices $ x $ and 
    $ y $ that are joind by at least two paths $ P_1, P_2$. As $ P_1 \neq P_2 $ there 
    is and edge $ e_0 $ that lies in $ P_1 $ but not in $ P_2$. We shall show that 
    the graph $ G - e_0 $ is still connected, which will be a contradiction.

    \smallskip \noindent
    Idea is to consider the path between any two vertices $ u,v $ and remove remove 
    the edge $ e_0 $ and still show we can reach $ v $ from $ v $ by using 
    the vertices $ x $ and $ y $ which have two paths between them and simply put 
    $ e_0 $ in one of them, then we can user the other one by going from one endpoint 
    from $ e_0 $ to $ x $, possible because $ G $ is connected. Then to $ y $ using the 
    other path and then from $ y $ to the other endpoint of $ e_0 $ and finally to $ v $.
    Thus $ u $ and $ v $ are still connected after removing $ e_0 $ a contradiction. 

\end{proof}

\paragraph{problem 6}
Let $ T_1,...,T_k $ be subtrees of a tree $ T $, any two of which have at least 
one vertex in common. Prove that there is a vertex common to all $ T_i$.
\begin{proof}
   Apply induction on $ |T| $. If $ |T| = 1 $ then $ T $ consists of a single vertex,
   say $ v$. If $ T_1,...,T_k $ are subtrees of $ T $ with pairwise intersecting
   vertex sets, then we must have $ T_i = T $ for each $ i = 1,...,k$. It follows 
   that $ v $ belongs to each $ T_i$.
   
   \smallskip \noindent
   So assume $ n \geq $ is an integer, and let us suppose this result holds for 
   all trees of order $ n - 1 $. Suppose $ T $ is a tree with $ |T| = n $ and 
   $ T_i,...,T_k $ are subtrees of $ T $. We assume $ k \geq 2 $. Let $ v $ be a 
   leaf in $ T $ and let $ T^\prime = T - v $ be the tree resulting from removing
   this leaf. 
\end{proof}

\paragraph{problem 7}