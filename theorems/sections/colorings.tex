\section{Colorings}
Note that a $k$-coloring is nothing but a vertex partition into 
$k$ independent sets, now called $ color classes$; the non-trivial
2-colourable graphs, are precisely the bipartite graphs.
The chromatic number is a key parameter in many extremal question, therefore
it is studied a lot.

\paragraph{Definitions}
\begin{itemize}
    \item \textit{vertex coloring} of a Graph $ G $ is a map 
    $c: V \to S $ such that $ c(v) \neq c(w) $ whenever $ v $ and 
    $ w $ are adjacent. The elements of the set $ S $ are called 
    available colors. The minimal $ k = |S| $ such that $ G $ has a
    $k$-coloring is the \textit{chromatic number} of $ G $, denoted 
    by $ \chi(G) $. A graph with $ \chi(G) = k $ is called $k$-chromatic.

    \item \text{edge coloring} is a map $c: E \to S $ with $ c(e) \neq c(f) $
    for any adjacent edges $ e,f $. We say the \textit{edge-chromatic number},
    or \textit{chromatic index} of $ G $ is the minimal $ k $ for which 
    a $k$-edge-coloring exists, it is denoted by . 
\end{itemize}

\paragraph{Relationship of $\chi(G)$ and $\chi^\prime(G)$}
Every edge coloring of $ G $ is a vertex coloring of its line graph
$ L(G) $, and vice versa, in particular $ \chi^\prime(G) = \chi(L(G))$.
The problem of finding good edge colorings may thus be viewed as a 
restriction of the more general vertex coloring problem to this special 
class of graphs. There are only very rough estimates for $\chi$ but 
$\chi^\prime$ always takes one ot two values, either $ \Delta $ or 
$ \Delta + 1$

\paragraph{Definitions}
\begin{itemize}
    \item $clique$ $number$ $\omega(G)$ of $G$ is the largest order of a clique in $G$
    \item $co$-$clique$ $number$ $\alpha(G)$ of $G$ is the largest order of an independent set in $G$
    \item  Graph is \textit{perfect} if $\chi(H) = \omega(H)$ for each induced
    subgraph $H$ of $G$. For example bipartite graphs are perfect with $\chi=\omega=2$
\end{itemize}

\paragraph{Lemma 46 (Simple Coloring Results)} $ $ \\
For any graph $G$ the following hold: 
\begin{itemize}
    \item $\chi(G) \geq max\{\omega(G), \frac{|G|}{\alpha(G)}\} $
    \item $||G|| \geq \binom{\chi(G)}{2}$ and $ \chi(G) \leq 1/2 + 
    \sqrt{2||G||+1/4}$
\end{itemize}

\paragraph{Greedy coloring algorithm}
\begin{proof}
     {\color{red}{TODO}}
\end{proof}

\paragraph{Lovasz Perfect Graph Theorem}
A graph $G$ is perfect if and only if its complement $\overline{G}$ is perfect.

\paragraph{Strong Perfect Graph Theorem} A graph $G$ is pefect if and only 
if it does not contain an odd cycle on at least 5 vertics (an \textit{odd hole})
or the complement of an odd hole as an induced subgraph.

\paragraph{Graphs with $\omega(G) \leq \chi(G)$ exist}
Meaning $ \exists G: \omega << \chi(G)$, we have 3 proofs 
\begin{itemize}
    \item Mycielski's construction: {\color{red}{TODO}}
    \item Tutte's construction: {\color{red}{TODO}}
    \item Erd\H{o}s-Hajnal theorem: girth greater than $k$ and chromatic 
    number greater than $k$
\end{itemize}

\paragraph{Theorem 5.1 (Brook's Theorem)} Let $ G $ be a connected graph.
Then $ \chi(G) \leq \Delta(G) $ unless $G$ is a complete graph or and odd cycle.
\begin{proof}
    {\color{red}{TODO}}
\end{proof}

\paragraph{Theorem 54 (K\H{o}nig, 1916)} If $ G $ is a bipartite graph
with maximum degree $\Delta$ then $\chi^\prime(G) = \Delta$.
\begin{proof}
    {\color{red}{TODO}}
\end{proof}

\paragraph{Theorem 5.4 (Vizing's Theorem)}
For any graph $ G $ with maximum degree $\Delta$
$$ \Delta \leq \chi^\prime(G) \leq \Delta + 1 $$
\begin{proof}
    {\color{red}{TODO}}
\end{proof}

\paragraph{Lemma 55} 
The list chromatic number of $ G = K_{n,n} $, with $n = \binom{2k}{k}$ 
is at least $k+1$.
\\
This means there exists a graph which has much greater list-chromatikc 
number than chromatic number
