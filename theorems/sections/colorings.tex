\section{Colorings}
Note that a $k$-coloring is nothing but a vertex partition into 
$k$ independent sets, now called $ color classes$; the non-trivial
2-colourable graphs, are precisely the bipartite graphs.
The chromatic number is a key parameter in many extremal question, therefore
it is studied a lot.

\paragraph{Definitions}
\begin{itemize}
    \item \textit{vertex coloring} of a Graph $ G $ is a map 
    $c: V \to S $ such that $ c(v) \neq c(w) $ whenever $ v $ and 
    $ w $ are adjacent. The elements of the set $ S $ are called 
    available colors. The minimal $ k = |S| $ such that $ G $ has a
    $k$-coloring is the \textit{chromatic number} of $ G $, denoted 
    by $ \chi(G) $. A graph with $ \chi(G) = k $ is called $k$-chromatic.

    \item \text{edge coloring} is a map $c: E \to S $ with $ c(e) \neq c(f) $
    for any adjacent edges $ e,f $. We say the \textit{edge-chromatic number},
    or \textit{chromatic index} of $ G $ is the minimal $ k $ for which 
    a $k$-edge-coloring exists, it is denoted by . 
\end{itemize}

\paragraph{Relationship of $\chi(G)$ and $\chi^\prime(G)$}
Every edge coloring of $ G $ is a vertex coloring of its line graph
$ L(G) $, and vice versa, in particular $ \chi^\prime(G) = \chi(L(G))$.
The problem of finding good edge colorings may thus be viewed as a 
restriction of the more general vertex coloring problem to this special 
class of graphs. There are only very rough estimates for $\chi$ but 
$\chi^\prime$ always takes one ot two values, either $ \Delta $ or 
$ \Delta + 1$

\paragraph{Definitions}
\begin{itemize}
    \item $clique$ $number$ $\omega(G)$ of $G$ is the largest 
    order of a clique in $G$

    \item $co$-$clique$ $number$ $\alpha(G)$ of $G$ is the largest order of an 
    independent set in $G$

    \item  Graph is \textit{perfect} if $\chi(H) = \omega(H)$ for each induced
    subgraph $H$ of $G$. For example bipartite graphs are perfect with
    $\chi=\omega=2$
\end{itemize}

\paragraph{Lemma 43} For any connected graph $G$ and for any vertex $v$ there
is an ordering of the vertices of $G: v_1,...,v_n$ such that $v=v_n$ and 
for each $i, 1 \leq i < n$, $v_i$ has a higher indexed neighbor.
\begin{proof}
    Consider a spanning tree $T$ of $G$ and create a sequence of sets 
    $X_1,...,X_{n-1}$ with $X_1 = V(G_1), X_i = X_{i-1} - \{v_{i-1}\}$, where
    $v_i$ is a leaf of $T[X_i]$ not equal to $v$, for $i = 1,...,n -1$.
    Then $v_1,...,v_n$ is a desired ordering with $v_n = v$
\end{proof}

\paragraph{Lemma 45} Let $G$ be a 2-connected non-complete graph of minimum
degree at least 3. Then there are vertices $x,y$ and $v$ such that 
$xy \notin E(G), xv,yv \in E(G)$ and $G-\{x,y\}$ is connected.
\begin{proof}
Consider a vertex $w$ of degree at most $|G|-2$

\smallskip \noindent
Case 1: $G-w$ has no cutvertices. \\
Let $x = w$ and $y$ be a vertex at distance 2 from $x$ and let $v$ be a 
common neighbor of $x$ and $y$. Since $y$ is not a cut-vertex in $G-x$, 
$G-\{x,y\}$ is connected.

\smallskip \noindent
Case 2: $G-w$ has a cutvertex. \\
In this case, let $v=w$. Then $v$ must be adjacent to non-cutvertex members of 
each leaf-block of $G-v$. Let $x$ and $y$ be such neighbors in distinct 
leaf-blocks. Since $v$ has another neighbor besides $x$ and $y$, 
$G-\{x,y\}$ is connected.
\end{proof}

\paragraph{Lemma 46 (Simple Coloring Results)} $ $ \\
For any graph $G$ the following hold: 
\begin{itemize}
    \item $\chi(G) \geq max\{\omega(G), \frac{|G|}{\alpha(G)}\} $
    \item $||G|| \geq \binom{\chi(G)}{2}$ and $ \chi(G) \leq 1/2 + 
    \sqrt{2||G||+1/4}$
\end{itemize}

\paragraph{Greedy coloring algorithm} $ $ \\
starting from a fixed vertex enumeration $v_1,...,v_n$ of $G$, we consider the 
vertices in turn and color each $v_i$ with the first available color, meaning 
with the smallest positive integer not already used to color any neighbour of 
$v_i$ among $v_1,...,v_{i-1}$. In this way we never use more than $\Delta(G) +1$
colors, even for unfavourable choices of the enumeration. If $G$ is complete or 
an odd cycle, then this is even best possible.

\paragraph{Lovasz Perfect Graph Theorem}
A graph $G$ is perfect if and only if its complement $\overline{G}$ is perfect.

\paragraph{Strong Perfect Graph Theorem} A graph $G$ is pefect if and only 
if it does not contain an odd cycle on at least 5 vertics (an \textit{odd hole})
or the complement of an odd hole as an induced subgraph.

\paragraph{Graphs with $\omega(G) \leq \chi(G)$ exist}
Meaning $ \exists G: \omega << \chi(G)$, we have 3 proofs 
\begin{itemize}
    \item Mycielski's construction 
    \item Tutte's construction
    \item Erd\H{o}s-Hajnal theorem: girth greater than $k$ and chromatic 
    number greater than $k$
\end{itemize}

\paragraph{Theorem 5.1 (Brook's Theorem)} Let $ G $ be a connected graph.
Then $ \chi(G) \leq \Delta(G) $ unless $G$ is a complete graph or and odd cycle.
\begin{proof}
Induction on $|V(G)| = n$. The theorem holds for any graph on at most 3
vertices. Assume $|V(G)| > 3$, we have two cases:

\bigskip \noindent
case 1: $G$ has a cut-vertex $v$ \\
If $G$ has a cut-vertex $v$, we can apply induction to the graphs $G_1$ and 
$G_2$ such that $G_1 \cup G_2 = G$ and $V(G_1) \cap V(G_2) = \{v\}$ and 
$|G_1| < |G|$ and $|G_2| < G$. By I.H. we get that even if $G_1$ and 
$G_2$ are complete or an odd cycle $\chi{G_i} \leq \Delta(G_i) \leq 
\Delta(G)$ we can make sure that the color of $v$ is the same in both 
colorings and still get $\chi(G) \leq \Delta(G)$.

\smallskip \noindent
If $\Delta(G) \leq 2$ then $G$ is a path or a cycle and the theorem holds.
Assume $\Delta(G) \geq 3$

\bigskip \noindent
case 2: $G$ is 2-connected.

\smallskip \noindent
case 2.1: There is a vertex $v$ of degree at most $\Delta -1$ \\
We shall order the vertices of $G$ $v_1,...,v_n$ such that $v = v_n$ and 
each $v_i, i<n$ has a neighbor with larger index, such an ordering exists 
by Lemma 43. Color $G$ greedily with respect to this ordering. We see at 
step $i$, there are at most $\Delta -1$ neighbors of $v_i$ that has been 
colored, so there is an available color for $v_i$

\smallskip \noindent
case 2.2: All vertices of $G$ have degree $\Delta$ \\
Consider vertices $x,y,v$ guaranteed by Lemma 45, such that $xy \notin E(G)$ 
and $xv,yv \in E(G)$ and $G-\{x,y\}$ is connected. Order the vertices 
of $G$ as $v_1,...,v_n$ such that $v_1 = x, v_2 = y, v_n = v$ for each 
$v_i, 3 \leq i < n$ there is a neighbor of $v_i$ with a higher index, such 
an ordering exists by Lemma 43. Color $G$ greedily according to this ordering. 
We see that $v_1$ and $v_2$ get the same color as in the previous case, at 
step $i, 3 \leq i < n$, $v_i$ has at most $\Delta -1$ colored neighbors so it 
could be colored with a remaining color. At the last step, we see that $v_n$
has $\Delta$ colored neighbors, but two of them $v_1$ and $v_2$ have the same 
color so there are at most $\Delta -1$ colors used by the neighbors of $v_n$.
Thus $v_n$ can be colored with a remaining color.
\end{proof}

\paragraph{Graphs with arbitrarily high chromatic number}
\paragraph{Mycielski's Construction} We can construct a family 
$(G_k = (V_k,E_k))_{k\in\mathbb{N}}$ of triangle-free graphs with 
$\chi(G_k) = k$ as follows:
\begin{itemize}
    \item $G_1$ is the single-vertex graph, $G_2$ is the single-edge $K_2$

    \item $V_{k+1} := V_k \cup U \cup \{w\}$ where 
    $V_k \cap (U \cup \{w\}) = \emptyset, V_k = \{v_1,...,v_n\}$ and 
    $U = \{u_1,...,u_n\}$

    \item $E_{k+1} := E_k \cup \{wu_i : i=1,...,k\} \cup 
    \bigcup^n_{i=1} \{u_iv: v \in N_{G_k}(v_i)\}$
\end{itemize}

\paragraph{Lemma} For any $k \geq 1$, Miycielsk's graph $G_k$ has chromatic 
number $k$. Moreover, $G_k$ is triangle-free.
\begin{proof}
    We shall prove this statement by induction on $k$ with trivial basis $k=1$.
    Assume $k \geq 2$ and $\chi(G_{k-1} = k-1$ and $G_{k-1}$ is triangle-free.

    \bigskip
    First we show that $\chi(G_k) = k$: \\
    We see that $\chi(G) \leq k$ by considering a proper coloring $c$ of 
    $G_{k-1}$ with colors from $[k-1]$ and letting $c': V_k \to [k]$ such
    that every $u_i$ gets the same color as $v_i$ and the vertex $w$ gets 
    the new color $k$.\\
    Now assume that $\chi(G_k) < k$. Let $c$ be a proper coloring of $G_k$ with
    colors from $[k-1]$. Derive contradiction if $\chi(G_k) = k-1$ by coloring
    $w$ with color $k-1$ and the vertices from $U_{k-1}$ are colored using 
    $k-2$ colors, but then the vertices in $V_{k-1}$ could also be colored
    from $[k-2]$.

    \bigskip
    To see that $G_k$ has no triangle, observe that a triangle could only have 
    one vertex in $u_i \in U_{k-1}$ and two vertices in $v_j,v_m \in V_{k-1}$.
    Then $v_i,v_j,v_m$ form a triangle in $G_{k-1}$ a contradiction.
\end{proof}

\paragraph{Tutte's Construction}
We can construct a family $(G_k)_{k\in\mathbb{N}}$ of triangle-free graphs 
with $\chi(G_k) = k$ as follows: $G_1$ is the single-vertex graph. To get 
from $G_k$ to $G_{k+1}$, take an independent set $U$ of size 
$k(|G_k| -1) +1$ and $\binom{|U|}{|G_k|}$ vertex-disjoint copies of $G_k$. 
For each subset of size $|G_k|$ in $U$ then introduce a perfect matching to 
exactly one of the copies of $G_k$.

\paragraph{Lemma} For any $k$, Tutte's graph $G_k$ has chromatic number $k$ and
it is triangle-free.
\begin{proof}
    We argue by induction on $k$ with trivial basis $k=1$. We see that 
    $\chi(G_k) \leq \chi(G_{k_1})+1$ because we can assign the same set of 
    $\chi(G_{k-1})$ colors to each copy of $G_{k-1}$ and a new color to $U$.
    Assume that $\chi(G_k) \leq \chi(G_{k-1})$. Consider a coloring of $G_k$
    with $\chi(G_{k-1})$ colors. By pigeonhole principle there is a set $U'$
    of $|G_{k-1}|$ vertices in $U$ of the same color, say 1. The vertices of 
    $U'$ are matched to a copy $G'$ of $G_{k-1}$. Then $G'$ does not use color 
    1 on its vertices and thus colored with less than $\chi(G_{k-1})$ colors.
    Therefore there are two adjacent vertices of the same color. So, any 
    proper coloring of $G_k$ uses more than $\chi(G_{k-1})$ colors.

    \bigskip
    To see that $G_k$ has no triangles, observe that any two adjacent edges 
    incident to $U$ have endpoints in distinct copies of $G_{k-1}$, thus are 
    not part of any triangle.
\end{proof}


\paragraph{Theorem 54 (K\H{o}nig, 1916)} If $ G $ is a bipartite graph
with maximum degree $\Delta$ then $\chi^\prime(G) = \Delta$.
\begin{proof}
    We see, that $\chi'(G) \geq \Delta$ because the edges incident to a vertex
    of maximum degree require distinct colors in a proper edge-coloring. 

    \bigskip
    To prove that $\chi'(G) \leq \Delta$ we use induction on $||G||$ with 
    a basis $||G|| =1$. Let $G$ be given, $||G|| \geq 2$ and assume 
    that the statement is true for any graph on at most $||G||-1$ edges.
    Let $e=xy \in E(G)$. By induction, there is a proper edge coloring $c$
    of $G' = G-e$ using $\Delta$ colors.

    \smallskip
    In $G'$ both $x$ and $y$ are incident to at most $\Delta-1$ edges. Thus
    there are non-empty color sets $Mis(y), Mis(y) \subseteq [\Delta]$, which
    are the colors that are not used on edges incident to $x$ or $y$.
    If $Mis(x) \cap Mis(y) \neq \emptyset$, color $e$ with 
    $\alpha \in Mis(x) \cap Mis(y)$. This gives $\chi'(G) \leq \Delta$.

    \smallskip
    If $Mis(x) \cap Mis(y) = \emptyset$, let $a \in Mis(x)$ and $b \in Mis(y)$,
    consider the longest path $P$ colored $a$ and $b$ starting at $x$. Because
    of parity, $P$ does not end in $y$, and because $y$ is not incident to 
    $b$, $y$ is not a vertex on $P$. Switch colors $a$ and $b$ on $P$. 
    Then we obtain a proper edge-coloring in which $b \in Mis(x) \cap Mis(y)$,
    which allows $e$ to be colored $b$. Thus $\chi'(G) \leq \Delta$.
\end{proof}
Proof sketch: induction: remove an edge, apply IH if the vertices incident to 
the edge have same color available then put back the edge and color it with it.
If they have different ones then consider longest path (alternating between
the available colors) starting in one of the 
vts, and switch colors of that one. So the vertices have the same color 
available, now put back the edge and color it with that color.

\paragraph{Theorem 5.4 (Vizing's Theorem)}
For any graph $ G $ with maximum degree $\Delta$
$$ \Delta \leq \chi^\prime(G) \leq \Delta + 1 $$
\begin{proof}
    proof sketch: 
    \begin{itemize}
        \item lower bound: because of vertex of maximum degree

        \item upper bound: induction on number of edges

        \item \textbf{Assume that $G$ has no proper edge-coloring with 
        max degree + 1 colors}

        \item introduce claim that when removing an edge we have 
        a path between the incident vertices in the two colors that are
        available at them 

        \item proof the claim with contradiction (we could switch colors and 
        obtain coloring with less colors)

        \item get contradiction using claim and creating multiple Paths, this
        means $G$ has a proper edge-coloring with $\Delta +1$
    \end{itemize}


    \bigskip
    The lower bound holds because the edges incident to a vertex of maximum
    degree require distinct colors in a proper edge-coloring. For the upper 
    bound use induction on $||G||$ with the trivial basis $||G|| = 1$. Let 
    $G$ be a graph, $||G|| > 1$, assume that the assertion holds for all 
    graphs with smaller number of edges. For any edge-coloring $c$ of a 
    subgraph $H$ of $G$ with colors $[\Delta+1]$.
    Assume now that $G$ has no proper edge-coloring with $\Delta+1$ colors.

    \smallskip
    \textbf{Claim} For any $e = xy \in E(G)$, for any proper coloring $c$ of 
    $G-e$ from $[\Delta +1]$, for any $\alpha \in Mis_c(x)$ and any 
    $\beta \in Mis_c(y)$, there is an $x$-$y$-path colored $\alpha$ and $\beta$.

    \smallskip
    We see that $Mis_c \neq \emptyset$ for any $v$. If $Mis_c(x) \cap Mis_c(y)
    \neq \emptyset$, let $\alpha \in Mis_c(x) \cap Mis_c(y)$. Color $xy$ with
    $\alpha$, this gives a proper coloring of $G$ with at most $\Delta +1$ 
    colors, a contradiction. 

    \smallskip 
    If $Mis_c(x) \cap Mis_c(y) = \emptyset$, let $\alpha \in Mis_c(x), \beta
    \in Mis_c(y), \alpha \neq \beta$. If there is a maximal path $P$ colored
    $\alpha$ and $\beta$ that contains $x$ and does not contain $y$, switch the 
    colors $\alpha$ and $\beta$ in $P$ and color $xy$ with $\beta$. This gives
    a proper coloring of $G$ with at most $\Delta+1$ colors, a contradiction,
    this proves the claim.

    \bigskip
    Let $xy_0 \in E(G)$. Let $c_0$ be a proper coloring of $G_0 := G - xy_0$
    from $[\Delta + 1]$. Let $\alpha \in Mis_{c_0}(x)$. Let $y_0,y_1,...,y_k$
    be a maximal sequence of distinct neighbors of $x$ such that 
    $c_0(xy_{i+1}) \in Mis_{c_0}(y_i), 0 \leq i < k$.

    \bigskip
    Let $c_i$ be a coloring of $G_i = G-xy_i$ such that 

    $$c_i(xy_j) = 
        \begin{cases}
            c_0(xy_{j+1}), \text{ for } j \in \{0,...,i-1\} \\
            c_0(e), \text{ otherwise}
        \end{cases}
    $$
    Note that $Mis_{c_i}(x) = Mis_{c_j}(x)$ for all $i,j \in \{0,...,k\}$.

    \smallskip
    Let $\beta \in Mis_{c_0}(y_k)$. Let $y = y_i$ be a vertex so that 
    $c_0(yx) = \beta$. Such a vertex exists, otherwise either 
    $\beta \in Mis_{c_k}(y_k) \cap Mis_{c_k}(x)$ contradicting Claim, or the 
    sequence $y_0,...,y_k$ can be extenden, contradicting its maximality.

    \smallskip
    Then $G_k$ has an $\alpha$-$\beta$ path $P$ with endpoints $y_{i-1},y_k$
    in $G_k -x$. On the other hand $G_i$ has an $\alpha$-$\beta$-path $P'$
    with endpoints $y_{i-1},y_i$ in $G_i -x$. Since $G_x$ is colored identically
    in $c_k$ and $c_i$, we have that $P \cup P'$ is a two-colored graph,
    connected since both paths contain $y_{i-1}$ and having three vertices 
    of degree 1. This is impossible.
\end{proof}

\paragraph{Lemma 55} 
The list chromatic number of $ G = K_{n,n} $, with $n = \binom{2k}{k}$ 
is at least $k+1$.
\\
This means there exists a graph which has much greater list-chromatic 
number than chromatic number
