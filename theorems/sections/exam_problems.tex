\section{Exam Problems}

\subsection{summer 18}



\subsection{winter 17/18}



\subsection{summer 16}



\subsection{winter 15/16}



\subsection{winter 12/13}

\paragraph{problem 1}
    Let $d = (d_1,...,d_n)$ be a sequence of $n \geq 2$ positive integers
    with $\sum_{i=1}^n d_i = 2n - 2$. Show that there exists an $n$-vertex
    tree with degree sequence $d$.

    \smallskip \noindent
    Solution: Proof by induction on $n:$ W.l.o.g. $d$ is sorted a descending 
    order. $n = 2$: because each element is positive and their sum is 2, 
    we have $d = (1,1)$. $K_2$ is a tree with this degree sequence.
    $n \geq 3:$ because of the sum condition, there is at least one element of 
    the sequence $d_j \geq 2$ and the last element $d_n$ is 1.
    The sequence $d' = (d_1,...,d_j -1,...,d_{n-1}$ fulfills the I.H., so there
    is a tree $T'$ with vertices $v_1,...,v_{n-1}$ of degrees 
    $d'_1,...,d'_{n-1}$. Then $T := T' + v_jv_n$ is a tree with $n$ vertices
    and degree sequence $d$.

\paragraph{problem 2}                           
    Show that every 2-connected graph, which is not an odd cycle, contains an
    even cycle.

    \smallskip \noindent
    Solution: Let $G$ be a 2-connected graph. Then $G$ has an ear-decomposition,
    so there are graphs $G_1,...,G_k = G$ such that $G_1$ is a cycle and 
    $G_{i+1}$ is constructed from $G_i$ by adding a path with distinct 
    endpoints in $G_i$, but not intersecting $G_i$ otherwise.

    \smallskip \noindent
    If $G_1$ is an even cycle, we are done. Otherwise $k$ is at least 2, since
    $G$ is not an odd cycle. Let $u,v$ be the endpoints of the path $P$ used 
    to construct $G_2$. Then there are two paths $P_1,P_2$ connecting $u$ and 
    $v$ in $G_1$. Since $G_1$ is and odd cycle, w.l.o.g. $P_1$ is of even 
    length, and $P_2$ is odd. If $P$ is even, then $P_1 \cup P$ is an even 
    cycle, otherwise $P_2 \cup P$ is an even cycle.


\paragraph{problem 3}
    Let $G$ be a bipartite graph with parts $A$ and $B$ and let $S$ be the set 
    of vertices of maximum degree in $G$. Prove that there exists a matching
    $M$ of $G$ covering every vertex in $A \cap S$.

    \smallskip \noindent
    Solution: Let $A' \subseteq A \cap S$. Since all vertices in $S$ are of 
    maximum degree, in $N(A')$ there are $k|A'|$ edges arriving from $A'$. 
    If $|N(A')| < |A'|$, then some vertex in $N(A')$ would need to have 
    degree $> k$, a contradiction to the maximum degree. So for all 
    $A' \subseteq A \cap S$, we have $|N(A')| \geq |A'|$. Thus the subgraph
    induced by $ (A \cap S) \cup B$ has a perfect matching by Hall's thm.


\paragraph{problem 4}
    Show that the maximum number of triangles in an $n$-vertex outerplanar 
    graph equals $n-2$, provided $n \geq 3$.  

    \smallskip \noindent
    Solution: Proof by induction on $n$, any graph on 3 vts contains at most 
    one triangle, so let $n \geq 4$ and $G$ be a outerplanar graph on $n$ vts.

    \smallskip \noindent
    Claim: $G$ contains a vertex $v$ of degree $\leq$ 2 \\
    If $G$ is not maximally outerplanar, add edges until this is the case. 
    Then the outer face is a cycle and every inner face is a triangle. We thus
    can find an edge on the inside whose vertices are of distance 2 on the 
    cycle (otherwise we could add another edge), the middle vertex on this 
    path is of degree 2. Now $G -v$ is outerplanar with at most $n-3$ 
    triangles (I.H.). By removing $v$, we removed at most one triangle, 
    so $G$ has at most $n-2$ triangles, thus proving the upper bound.

    \smallskip \noindent
    We give a construction for the lower bound: \\
    Let $G$ be outerplanar and $uv$ an edge on the outer face.
    Add a new vertex $w$ on the outside of $G$ and add the edges $uw,vw$ to $G$,
    thus creating a new triangle, but not removing any vertices from the outer 
    face.


\paragraph{problem 7}
    Prove that almost all graphs $G$ in $\mathcal{G}(n,\frac{1}{2})$ fulfill
    $\chi(G) \geq \sqrt{n}$

    \smallskip \noindent
    Solution: Since each color class in a graph froms an independent set, the 
    chromatic number of an $n$-vertex graph $G$ can be bounded by 
    $\chi(G) \geq \frac{n}{\alpha(G)}$. \\
    For a random graph $G \in \mathcal{G}(n,\frac{1}{2})$ we thus have
    $$
        \text{P}(\chi(G) \leq \sqrt{n}) \leq \text{P}(\alpha(G) \geq \sqrt{n})
        \leq \binom{n}{\sqrt{n}} 2^{- \binom{\sqrt{n}}{2}} 
        \leq n^{\sqrt{n}} 2^{\sqrt{n} - n} 
        = \frac{(2n)^{\sqrt{n}}}{2^n} \underset{n \to \infty}{\rightarrow} 0
    $$

\paragraph{problem 8}
    Prove that for $n \geq 5$, every triangle-free non-bipartite graph on $n$ 
    vertices contains at most $\lfloor\frac{(n-1)^2}{4}\rfloor +1$ edges.

    \smallskip \noindent
    Solution: PRoof by contradiction: Assume we have a edge-maximal 
    triangle-free non-bipartite graph $G$ on $n$ vertices and at least 
    $\lfloor\frac{(n-1)^2}{4}\rfloor + 2$ edges. Since $G$ is non-bipartite,
    it contains an odd cycle. Consider a shortest odd cycle $C$.

    \smallskip \noindent
    Claim 1: $C$ is of length 5. \\
    Assume $C$ was of length $\geq 7$, then we could split the cycle into an
    even path $P_1$ of length 4 and an odd path $P_2$ of length $\geq 3$. 
    Denote the endpoints of the paths by $u$ and $v$. Then $uv$ cannot be an 
    edge of $G$, because otherwise $P_1 + uv$ would be a shorter odd cycle.
    If $u$ and $v$ have a common neighbor $w$, then $P_2 + uw + wv$ would be 
    a shorter odd cycle. Adding $uv$ could thous not induce a triangle, so 
    $G +uv$ would still be triangle-free and non-bipartite, so $G$ could not 
    have been edge-maximal, contradicting our assumption.

    \smallskip \noindent
    Claim 2: Removing $C$ from $G$ removes at most $2n-5$ edges. \\
    Any vertex in $G-C$ can be connected to at most 2 vts of $C$, because
    otherwise, two neighboring vertices in $C$ would share a neighbor in $G-C$,
    thus inducing a triangle. We thus lose at most $2(n-5)$ edges outside $C$
    and 5 edges in $C$, which gives us the claim.

    \smallskip \noindent
    From these claims, we can derive that we can always remove a $C_5$ from $G$
    without losing more than $2n-5$ edges. We have $(n-1)^2/4 + 2 - (2n-5) 
    = (n-5)^2 / 4 + 1$ \\
    For odd $n$, the number of edges in $G-C$ is greater than 
    $t_2(n-5) = (n-5)^2/4$, so by Turan's theorem, $G-C$ contains a triangle,
    a contradiction. For even $n$ the number of edges in $G$ is 
    at least $\lceil\frac{(n-1)^2}{4}\rceil +1$, so $G-C$ contains at least 
    $\lceil(n-5)^2/4\rceil >  \lceil(n-5)/2\rceil \lfloor(n-5)/2\rfloor 
    = t_2(n-5)$ edges. By Turans theorem $G$ cannot be triangle free, a 
    contradiction.

    \smallskip
    We thus have shown that such a graph can have at most 
    $\lfloor(n-1)^2/4\rfloor + 1$ edges, otherwise it would not be 
    triangle-free.



\paragraph{problem 9}
    Prove that for every $k$ there exists an $n = n(k)$ such that any set of 
    $n$ points in the plane contains a subset of $k$ points that can be 
    covered with a disc of radius 1, or a subset of $k$ points which cannot 
    be covered with less than $k$ discs of radius 1 each.

    \smallskip \noindent
    Solution: For any set $P$ of $n$ points on the plane, we construct a 
    graph $G = (P,E)$ in which two points are connected by an edge if their 
    distance is at most 2. If this graph contains an independent set of 
    size $k$, the corresponding points cannot be covered by disks of radius 
    1 such that any two disk covers two points. If this graph contains a 
    clique of size $9k$, the corresponding points can be covered by a disk 
    of radius 2 around any point in the clique. This disk can be covered 
    by 9 disks of radius 1, so by pigeonhole principle ,there is a disk of 
    radius 1 that coveres $k$ of the points. If we set $n:= R(k,9k)$, for 
    any set of $n$ points we either find an independent set of size $k$
    or a clique of size $9k$ in the corresponding graph, thus fulfilling the 
    condition.


\paragraph{problem 10} $ $\\
    For every $k$ there exists an $m = m(k)$ such that every graph on $m$ 
    edges contains a vertex of degree at least $k$ or a matching on at least 
    $k$ edges. \\
    This is true: for $m = 2k^2$, if there is no vertex of degree $k$, 
    we need at least $2k$ vertices. If we remove an edge and its vts, we 
    lose at most $2(k-2)+1$ edges and 2 vertices. By succesively removing
    $k$ edges and their vertices, we lose at most $k(2k-3) < 2k^2$ edges 
    and $2k$ vertices, so in every step, there are edges left to be removed.
    The removed edges then form a matching of size $k$, since they cannot 
    share common vertices.

    \bigskip \noindent
    For every $k\geq 3$ there exists a $\chi = \chi(k)$ such that every graph
    $G$ \\ This statement is false, triangle-free graphs with arbitrary 
    chromatic number can be constructed via Tutte's or Mycielski's

   \bigskip \noindent 
   Any $n$-vertex graph with at least $\frac{n^2}{4}$ eges is Hamiltionian.\\
   This is false, the graph could have an isolated vertex.



\subsection{winter 11/12}
