\section{Extremal graph theory}
In this chapter we study how global parameters of a graph, such as its edge density 
or chromatic number, can influence its local substructures.
Two categories of questions:
\begin{itemize}
    \item global assumptions that might imply that $ H $ as a $minor$ (or topological minor)
    it will suffice to raise $ ||G|| $ above the value of some linear function 
    of $|G|$.
    \item global assumptions that might imply the existence of some given graph $ H $ 
    as  a $subgraph$ 
\end{itemize}

\paragraph{Definition} 
\begin{itemize}
    \item The \textit{extremal number} ex$(n,H)$ denotes the maximum
        size (amount of edges) of a graph of order $ n $ that does not contain
        $ H $ as a subgraph and EX$(n,H)$ is the set of $H$-free graphs on $n$
        vetices with ex$(n,H)$ edges.
    \item Example: ex$(n,P_3) = \lfloor\frac{n}{2}\rfloor$, EX$(n,P_3)
    = \{\lfloor n\rfloor \cdot K_2 + (n \text{ mod } 2) \cdot E_1\} $
    \item Let $ n $ and $ r $ be integers with $ 1 \leq r \leq n $ The 
    \textit{Turan graph} $T_r(n)$ is the unique complete $r$-partite graph of order 
    $n$ whose partite sets differ by at most 1 in size. It does not contain 
    $K_{r+1}$, we denote $||T_r(n)||$ by $t_r(n)$
\end{itemize}

\paragraph{Calculating Turan Number}
The Turan number $t_r(n)$ is calculated as follows:\\
Let $n = pr + s$ where $ p $ and $ s $ are integers, and 
$ 0 \leq s < r $, then $ T_r(n) $ has $ s $ parts of size $p + 1$
and $r - s$ parts of size $ p$. And then simply count edges by hand.

\paragraph{Lemma 58} For any $ r,n \geq 1, t_r(n+r) = t_r(n) + n(r-1) + \binom{r}{2}$
\begin{proof}
    Consider $ G = T_r(n+r) $ graph with parts $ V_1,...,V_r$. Let $ v_i \in V_i,
    i = 1,...,r$. Then $ G^\prime = G - \{v_1,...,v_r\} $ is isomporphic to $T_r(n)$.
    We have that $ ||G|| - ||G^\prime|| $ is equal to the number of edges incident to 
    $ v_i$'s for some $ i = 1,...,r$. This number is 
    $$ \underbrace{n(r-1)}_{\text{every vertex in } T_r(n) \text{ gets (r-1) new}} + 
    \underbrace{\binom{r}{2}}_\text{edges between new r vts} $$
\end{proof}

\paragraph{Lemma 59} Among all $n-$vertex $r$-partite graphs, $ T_r(n) $ has the largest 
number of edges.
\begin{proof}
    Let first $ r = 2 $: \\
    Let $ G $ be an $n$-vertex bipartite graph with largest possible
    number of edges. Then clearly $ G $ is complete bipartite. Assume that two parts $ V $
    and $ U $ of $ G $ differ in size by at least 2, so $ |V| > |U| + 1$. Put one 
    vertex from $ V $ to $ U $ to obtain new parts $ V^\prime $ and $ U^\prime $ 
    and let $ G^\prime $ be complete bipartite graph with parts $ V^\prime $ and 
    $ U^\prime $. Then $ ||G^\prime|| = |V^\prime||U^\prime| = (|V|-1)(|U|+1) 
    = |V||U| - |U| + |V| -1 > |V||U| - |U| + |U| + 1 - 1 = |V||U| = ||G||$, a 
    contradiction to maximality of $G$.

    \bigskip \noindent
    If $ r > 2 $: \\
    Consider any two parts $ U, V $ of an $r$-partite $G$. Assume that $ U $ differs
    from $ V $ by at least 2 in size. Let $ X $ be the remaining set of vertices. 
    Then $ ||G|| = ||G[X]|| + |X|(n-|X|) + ||G[U\cup V]|| $.  Let $ G^\prime $ be a 
    graph on the same set of vertices as $ G $ that differs from $ G $ only on edges 
    induced by $ U \cup V $ and so that $ G^\prime[U \cup V] $ is a balanced complete 
    bipartite graph. Then from the previous paragraph with $ r = 2 $, we see that 
    $ ||G^\prime[U \cup V]|| > ||G[U \cup V]||$. Thus $ ||G^\prime|| > ||G||$, a
    contradiction. Thus any two parts of $ G $ differ in size by at most 1. In 
    addition we see as before that $ G $ is complete $r$-partite. Thus $ G $ is 
    isomporphic to $T_r(n)$.      
\end{proof}

\paragraph{Lemma 60} For a fixed $ r $, 
$$ \lim_{n\to\infty} \frac{t_r(n)}{\binom{n}{2}} = 1 - \frac{1}{r}$$
\begin{proof}
    Since each part in $ T_r(n) $ has size either $\lfloor\frac{n}{r}\rfloor$
    or $\lceil\frac{n}{r}\rceil$, we see that each part has size between
    $ \frac{n-r}{r}$ and $ \frac{n+r}{r}$. We have that 
    $$ \underbrace{\binom{n}{2}}_{\text{all edges}} - 
    \underbrace{r \binom{(n+r)/r}{2}}_{r \text{ independent sets}} 
    \leq t_r(n) \leq \binom{n}{2} - r\binom{(n-r)/r}{2} $$
    Thus 
    $$ \binom{n}{2} - r \frac{1}{2}\frac{(n+r)}{r}\frac{n}{r}
    \leq t_r(n) \leq \binom{n}{2} - r \frac{1}{2}\frac{(n-r)}{r}\frac{(n-2r)}{r} 
    $$
    More manipulation and dividing by $ \binom{n}{2} $ gives the result.
\end{proof}

\paragraph{Theorem 62 (Mantel's theorem)} If a graph $ G $ on n vertices contains 
no triangle then it contains at most $ \frac{n^2}{4} $ edges.
\begin{proof} First proof: \\
    Suppose that $ G $ has $ m $ edges. Let $ x $ and $ y $ be two vertices 
    in $ G $ which are joined by an edge. We see that $ d(x) + d(y) \leq n $, because 
    every vertex in the graph $ G $ is connected to at most one of $ x $ and $ y $. 
    Note now that 
    $$ \sum_x d^2(x) = \sum_{xy\in E} (d(x) + d(y)) \leq mn $$
    On the other hand, since $ \sum_{x} d(x) = 2m$, the Cauchy-Schwarz 
    inequality implies that 
    $$ \sum_x d^2(x) \geq \frac{(\sum_x d(x))^2}{n} \geq \frac{4m^2}{n} $$
    Therefore
    $$ \frac{4m^2}{n} \leq mn $$ 
    an the result follows.
\end{proof}
\begin{proof} Second proof: \\
    We proceed by induction on $ n $. For $ n = 1 $ and $ n = 2$. The result 
    is trivial, so assume that $ n > 2 $ and we know it to be true for $ n - 1$.
    Let $ G $ be a graph on $ n $ vertices. Let $ x $ and $ y $ be two adjacent vertices
    in $ G $. Since every vertex in $ G $ is connected to at most one of $ x $ and $ y$, 
    there are at most $ n - 2 $ edges between $ \{x,y\} $ and $ V(G) - \{x,y\} $. 
    Let $ H = G - \{x,y\}$. Then $ H $ contains no triangles and thus, by induction, 
    $ H $ has at most $ \frac{(n - 2)^2}{4} $ edges. Therefore the total number 
    of edges in $ G $ is at most

\end{proof}

\paragraph{Theorem 6.2 (Turan's Theorem)} For all integers $ r > 1 $ and $ n \geq 1$,
any graph $ G $ with $ n $ vertices, ex$(n,K_r)$ edges and $ K_r \nsubseteq G $ 
is a $ T_{r-1}(n) $. In other words EX$(n,K_r) = \{T_{r-1}(n)\}$.
\begin{proof}
        
\end{proof}

\paragraph{Conjecture Erd\H{o}s-Sos} If $ |G| = n$ and $||G|| > \frac{(k-1)n}{2}$
, then $ G $ contains all $k$-edge trees as subgraphs, i.e. for any tree 
$ T $ on $ k $ edges ex$(n,T) \leq \frac{(k-1)n}{2}$

\paragraph{Theorem 66 (Erd\H{o}s-Stone-Simonovits)} For any graph $ H $ and for any 
fixed $ \epsilon > 0 $ there is $ n_0 $ such that for any $ n \geq n_0 $:
$$ (1 - \frac{1}{\chi(H)-1} - \epsilon)\binom{n}{2} \leq ex(n,H) \leq
(1 - \frac{1}{\chi(H)-1}+\epsilon) \binom{n}{2} $$

\smallskip \noindent
\textit{Proof outline:} Let $ r = \chi(H) - 1 $ \\
For the upper bound, let $ G $ be a graph on $ n $ vertices that has 
$ (1 - \frac{1}{\chi(H) - 1} + \epsilon) \binom{n}{2} $ edges. We shall show 
that $ G $ has a subgraph isomorphic to $ H $. Let $ G^\prime $ be a large 
subgraph of $ G $ that has minimum degree at least 
$ (1 - \frac{1}{r} + \frac{\epsilon}{2}) |V(G^\prime)|$, we can finde such 
$ G\prime $ by greedily deleting vertices of smaller degrees. Then show, 
by induction on $ r $ that $ G^\prime $ contains a complete $(r+1)$-partite 
graph $ H^\prime $ with sufficiently large parts. Finally observe that 
$ H \subseteq H^\prime $

\paragraph{Definition Zarankiewicz function} $ z(m,n;s,t) $ denotes the maximum
number of edges that a bipartite graph with parts $ x,Y$ of sizes $ m,n$
respectively, can have without containing $ K_{s,t} $ respecting sides
(i.e. there is no copy of $ K_{s,t} $ with partition sets $ S,T $ of sizes
$s,t $ respectively, such that $ S \subseteq X $ and $ T \subseteq Y $)

\paragraph{Theorem 67 (Kovari-Sos-Turan)} We have the upper bound 
$$ z(m,n;s,t) \leq (s - 1)^{\frac{1}{t}}(n-t+1)m^{1-\frac{1}{t}}
+ (t-1)m $$ 
In particular for $ m = n $ and $ t = s $
$$  z(m,n;s,t) \leq c_1 \cdot n \cdot n^{1-\frac{1}{t}} + c_2 \cdot n
= \mathcal{O}(n^{2-\frac{1}{t}}) $$ 
\begin{proof}
    Let $ G $ be a bipartite graph with parts $ A $, $ |A| = m$ and 
    $ B $, $|B| = n $ such that it does not contain a copy of $ K_{s,t} $
    with part of size $ s $ in $ A $ and part of size $ t $ in $ B $.
    Let $ T $ be the number of stars of size $ t $ with a center in $ A $
    Then 
    $$ T = \sum_{v \in A} \binom{deg(v)}{t} $$ 
    On the other hand 
    $$ T \leq (s - 1)\binom{n}{t} $$
    Since for each subset $ \mathcal{Q} $ of $ t $ vertices in $ B $ there 
    are at most $ s-1 $ stars counted by $ T $ with a leaf-set $ \mathcal{Q}$.

    \smallskip TODO
\end{proof}

\paragraph{Lemma 68} For any positive integers $ n,t $ with $ t < n$, 
$$ ex(n,K_{t,t}) \leq \frac{z(n,n;t,t)}{2} $$

\paragraph{Theorem 69} For any positive $ t$, and $ n > t $, there are 
positive constants $ c $ and $ c^\prime $ such that 
$$ c^\prime \cdot n^{2 -\frac{2}{t+1}} \leq ex(n,K_{t,t}) \leq c \cdot n^{2 - \frac{1}{t}} $$

\paragraph{Theorem 71} 
$$ ex(n,C_4) = \frac{1}{2}n^{3/2} + \mathcal{o}(n^{3/2}) $$ 
$$ ex(n,C_6) = \Theta(n^{4/3}) $$
$$ ex(n,C_10) = \Theta(n^{6/5}) $$ 
$$ c^\prime \cdot n^{1-\frac{2}{3k-2-\epsilon}} \leq 
ex(n,C_{2k}) \leq c \cdot n^{1 + \frac{1}{k}} $$ 

\paragraph{Definition 6.4} Let $ X,Y \subseteq V(G) $ be disjoint
vertex sets and $ \epsilon > 0 $. 
\begin{itemize}
    \item \textit{the density} $d(X,Y)$ of $(X,Y)$ is 
    $$ d(X,Y) := \frac{||X,Y||}{|X|,|Y|} $$ 
    \item For $ \epsilon > 0 $ the pair $ (X,Y)$ is an $\epsilon$-$regular$ $pair$
    if we have 
    $$ |d(X,Y) - d(A,B)| \leq \epsilon $$ 
    for all $ A \subseteq X, B \subseteq Y $ with 
    $|A| \geq \epsilon|X| $ and $ |B| \geq \epsilon|Y| $ \\
    In other words, the edges in an $\epsilon$-regular pair are distributed
    very uniformly, with the density between any pair of reasonably large 
    subsets of vertices being very close to the overall density of the pair.
    This uniform distribution of edges is typical in a random bipartite graph,
    and captures what we mean when we say a (bipartite) graph 'looks random'.
    It remains to define what kinds of partitions of the vertices we will 
    be concerned with.
    \item An $\epsilon$-$regular$ $partition$ of the graph $ G = (V,E) $ is 
    a partition of the vertex set $ V  = V_0 \cupdot ... \cupdot V_k $ with 
    the following properties:
        \begin{enumerate}
            \item $|V_0| \leq \epsilon|V| $
            \item $ |V_1| = |V_2| = ... = |V_k| $
            \item all but at most $ \epsilon k^2$ of the pairs $ (V_i,V_j) $
            for $ 1 \leq i < j \leq k $ are $\epsilon$-regular.
        \end{enumerate}
    Note the parameter $\epsilon$ play three roles here: bounding the size of 
    the exceptional set $ V_0 $, bounding the number of irregular pairs, and 
    controlling the regularity of the regular pairs
\end{itemize}
In an $\epsilon$-regular parition we have control over the distribution of edges 
between the $\epsilon$-regular pairs, but not over the edges within any of the 
parts, involving the exceptional set, or in irregular pairs. Thus, in light of 
the three roles described above, the smaller $\epsilon$ is, the greater our 
control over the distribution of edges in an $\epsilon$-regular partition.

\bigskip \noindent
Informally, the Regularity Lemma tells us that the vertices of $ any $ large 
graph can be partitioned into a bounded number of parts, with the subgraph 
between most pairs of parts looking random.

\paragraph{Szemeredi's Regularity Lemma} For any $\epsilon > 0 $ and any 
integer $ m \geq 1 $ there is an $ M \in \mathbb{N} $ such that every graph 
of order at least $ m $ has an $\epsilon$-regular partition 
$V_0 \cupdot...\cupdot V_k$ with $ m\leq k \leq M $.

\paragraph{Erdos-Stone Theorem} For all integers $ r > s \geq 1 $ and any 
$ \epsilon > 0 $ there exists an integer $ n_0 $ such that every graph with 
$n \geq n_0$ vertices and at least
$$ t_{r-1}(n) + \epsilon n^2 $$
edges contains $ K^s_r $ (is the complete $r$-partite graph where each 
part contains exactly $ s $ vertices) as a subgraph.

\paragraph{Corollary 73} Erd\H{o}s-Stone together with 
$ \lim_{n\to\infty} / \binom{n}{2} = 1 - 1 / r $ yields an asymptotic formula 
for the extremal number of any graph $ H $ on at least one edge:
$$ lim_{n\to\infty} \frac{ex(n,H)}{\binom{n}{2}} = \frac{\chi(H)-2}{\chi(H)-1} $$


