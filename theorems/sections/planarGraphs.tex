\section{Planar graphs}

\paragraph{planar graph vs. plane graph}
Plane Graph is topological object $ (V,E), V \subseteq \mathbb{R}^2$, 
$ e \in E $ are arcs in $ \mathbb{R}^2$.
Planar graph is combinatorial object $ (V,E), E \subseteq \binom{n}{2} $ s.t. 
it has a plane graph realization (planar embedding)

\paragraph{Definition outerplanar}
A graph $ G $ is $outerplanar$ if it has a plane embedding such that the
boundary of the outer face contains all vertices $ V $.


\paragraph{Theorem 32 (Plane triangulation)} A graph of order at least 3 is 
maximally plane if and only if it is a plane triangulation.
\begin{proof}
    {\color{red}{TODO}}    
\end{proof}

\paragraph{Theorem 4.2 (Euler's Formula)} Let $ G $ be a connected plane graph 
with $ n $ vertices, $ m $ edges and $ l $ faces. Then 
$$ n - m + l = 2 $$
\begin{proof}
    {\color{red}{TODO}}    
\end{proof}

\paragraph{Corollary 33} A plane graph with $ n \geq 3 $ vertices has at most 
$ 3n - 6 $ edges. Every plane triangulation has exactly $ 3n - 6 $ edges.
\begin{proof}
    {\color{red}{TODO}}   
\end{proof}

\paragraph{Theorem 4.4 (Kuratowski's Theorem)} The following statements are 
equivalent for graphs $ G$:
\begin{enumerate}
    \item $ G $ is planar 
    \item $ G $ does not have $ K_5 $ or $ K_{3,3} $ as minors
    \item $ G $ does not have $ K_5 $ or $ K_{3,3} $ as topological minors
\end{enumerate}

\paragraph{Theorem 4.7 (5-Color Theorem)} Every planar graph is 5-colorabel
\begin{proof}
     {\color{red}{TODO}}   
\end{proof}

\paragraph{Definition} List coloring
\begin{itemize}
    \item Let $L(v) \subseteq \mathbb{N}$ be a list of colors for each vertex 
    $v \in V$. We say that $ G $ is $L$-$list$-$colorable$ if there is a coloring
    $c: V \to \mathbb{N}$ such that $c(v) \in L(v)$ for each $v \in V$ and 
    adjacent vertices receive different colors. 
    \item Let $k \in \mathbb{N}$. We say that $G$ is $k$-$list$-$colorable$ or 
    $k$-$choosable$ if $G$ is $L$-list-colorable for each list $L$ with 
    $|L(v)| = k$ for all $v \in V$
    \item the $choosability$, denoted by ch$(G)$ is the smallest $k$ such that
    $G$ is $k$-choosable.
    \item the $edge$ $choosability$, denoted by $\text{ch}^\prime(G)$ is the 
    smallest $k$ such that $G$ is $L$-edge-list-colorable for each list
    $L$ with $|L(e)| = k$ for $e \in E$
\end{itemize}

\paragraph{Theorem 4.10 (5-List-Color Theorem)} Let $ G $ be a planar graph.
Then the list chromatic number of $ G $ is at most 5.
\begin{proof}
    {\color{red}{TODO}}
\end{proof}