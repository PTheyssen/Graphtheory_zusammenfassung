\section{Colorings}
Note that a $k$-coloring is nothing but a vertex partition into 
$k$ independent sets, now called $ color classes$; the non-trivial
2-colourable graphs, are precisely the bipartite graphs.

\paragraph{Definitions}
\begin{itemize}
    \item \textit{vertex coloring} of a Graph $ G $ is a map 
    $c: V \to S $ such that $ c(v) \neq c(w) $ whenever $ v $ and 
    $ w $ are adjacent. The elements of the set $ S $ are called 
    available colors. The minimal $ k = |S| $ such that $ G $ has a
    $k$-coloring is the \textit{chromatic number} of $ G $, denoted 
    by $ \chi(G) $. A graph with $ \chi(G) = k $ is called $k$-chromatic.
    \item \text{edge coloring} is a map $c: E \to S $ with $ c(e) \neq c(f) $
    for any adjacent edges $ e,f $. We say the \textit{edge-chromatic number},
    or \textit{chromatic index} of $ G $ is the minimal $ k $ for which 
    a $k$-edge-coloring exists, it is denoted by . 
\end{itemize}

\paragraph{Relationship of $\chi(G)$ and $\chi^\prime(G)$}
Every edge coloring of $ G $ is a vertex coloring of its line graph
$ L(G) $, and vice versa, in particular $ \chi^\prime(G) = \chi(L(G))$.
The problem of finding good edge colorings may thus be viewed as a 
restriction of the more general vertex coloring problem to this special 
class of graphs. There are only very rough estimates for $\chi$ but 
$\chi^\prime$ always takes one ot two values, either $ \Delta $ or 
$ \Delta + 1$