\section{Ramsey Theory}

\paragraph{Definitions}
\begin{itemize}
    \item In an edge-coloring of a graph, a set of edges is 
        \begin{itemize}
            \item $monochromatic$ if all edges have the same color
            \item $rainbow$ if no two edges have the same color
            \item $lexical$ if two edges have the same color iff
            they have the same lower endpoint in some ordering of 
            the vertices 
        \end{itemize}

    \item Let $k$ be a natural number. Then the \textit{Ramsey number}
    $R(k) \in \mathbb{N}$ is the smallest $n$ such that every 
    2-edge-coloring of $K_n$ contains a monochromatic $K_k$.

    \item \textbf{asymmetric Ramsey number $R(k,l)$:} is the smallest $n\in\mathbb{N}$
    such that every 2-edge-coloring of a $K_n$ contains a red $K_k$
    or a blue $K_l$. 

    \item \textbf{graph Ramsey number $R(G,H)$:} is the smallest $n\in\mathbb{N}$
    such that every red-blue edge-coloring of $K_n$ contains a red $G$
    or a blue $H$.

    \item \textbf{hypergraph Ramsey number $R_r(l_1,...,l_k)$:} is the smallest 
    $n\in\mathbb{N}$ such that every $k$-coloring of the edges of the 
    complete hypergraph on $n$ vertices and edges of size $r$ contains 
    a clique of size $l_i$ whose edges all have color $i$, for some 
    $ i \in \{1,...,k\}$.

    \item \textbf{induced Ramsey number $IR(G,H)$:} is the smallest $n \in \mathbb{N}$
    for which there is a graph $F$ on $n$ vertices such that in any red-blue coloring
    of $E(F)$, there is an induced subgraph of $F$ isomorphic to $G$ with all 
    its edgs colored red or there is an induced subgraph of $F$ isomorphic to 
    $H$ with all its edges colored blue.

    \item \textbf{anti-Ramsey number $AR(n,H)$:} is the maximum number of colors
    that an edge-coloring of $K_n$ can have without containing a rainbow copy 
    of $H$.
\end{itemize}


\paragraph{Ramsey Theorem} For any $k \in \mathbb{N}$ we have 
$$ \sqrt{2}^k \leq R(k) \leq 4^k $$ 
In particular the Ramsey numbers, the asymmetric Ramsey numbers an the graph 
Ramsey numbers are finite.
\begin{proof}
    {\color{red}{TODO}}
\end{proof}

\paragraph{Remark}
$R(2) = 2, R(3) = 6, R(4) = 18$ and $43 \leq R(5) \leq 48$.

\paragraph{Applications of Ramsey theory}

\paragraph{Theorem (Erd\H{o}s, Szekeres)} Any list of more than $n^2$ numbers 
contains a nondecreasing or non-increasing sublist of more than $n$ numbers.
\begin{proof}
    {\color{red}{TODO}}
\end{proof}

\paragraph{Theorem (Erd\H{o}s, Szekeres)} For any integer $m \geq 3$ there is 
an integer $N = N(m)$ such that if $X$ is a set of $N$ points on the plane 
such that no three points are on a line, then $X$ contains a vertex set of a convex 
$m$-gon.
\begin{proof}
    {\color{red}{TODO}}
\end{proof}

\paragraph{Definition}
Let $R(p,q;r)$ be the hypergraph Ramsey number for $r$-uniform hypergraphs.
The following Theorem show the existence of hypergraph Ramsey number.

\paragraph{Theorem 83} For any parametres $p,q,r \geq 2$
$$ R(p,q;r) \leq R(R(p-1,q;r), R(p,q-1;r);r-1) + 1 $$ 