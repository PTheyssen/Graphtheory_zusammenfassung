\section{Hamiltonian cycles}
\paragraph{Definition}
\begin{itemize}
    \item A cycle $ C $ in a graph $ G $ is $Hamiltonian$ 
    if it contains all vertices.

    \item  A graph that has a Hamiltonian cycle is called Hamiltonian graph.
\end{itemize}


\paragraph{Lemma 10.1 (Neccessary condition for existence of Ham. cycle)} $ $\\
If $G$ has a Hamiltonian cycle, the for every non-empty $S \subseteq V$ the 
graph $G - S$ cannot have more than $|S|$ components.
\begin{proof}
    Let $ C $ be a Hamiltonian cycle of $ G$.
    Let $ S \subseteq V(G) $, $ S \neq \emptyset $, $ t := $ \# components of $ G - S $
    There are at least 2 edges of $ C $ between each component of $ G - S $
    and $ S$. If $ e = \#$ edges of $ C $ between $ S $ and $ V - S$, 
    we have 
    $$ e \geq t \cdot 2 \text{ and } e \underset{C is 2-\text{regular}}{\leq}
    |S| \cdot 2 $$ 
\end{proof}

\paragraph{Theorem 10.2 Dirac} Every graph with $n \geq 3$ vertices and 
minimum degree at least $\frac{n}{2}$ has a Hamiltonian cycle.
\begin{proof}
    {\color{red}{TODO}}
\end{proof}

\paragraph{Ore's Thm.}
A graph $ G $ on $ n \geq 3 $ vertices is Hamiltonian if  
$ \forall u,v \in V(G), uv \notin E(G), d(u)+d(v) \geq n $