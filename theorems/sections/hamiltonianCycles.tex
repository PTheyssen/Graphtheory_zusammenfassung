\section{Hamiltonian cycles}
\paragraph{Definition}
\begin{itemize}
    \item A cycle $C$ in a graph $G$ is $Hamiltonian$ 
    if it contains all vertices.

    \item  A graph that has a Hamiltonian cycle is called Hamiltonian graph.
\end{itemize}


\paragraph{Lemma 10.1 (Neccessary condition for existence of Ham. cycle)} $ $\\
If $G$ has a Hamiltonian cycle, the for every non-empty $S \subseteq V$ the 
graph $G - S$ cannot have more than $|S|$ components.
\begin{proof}
    Let $C$ be a Hamiltonian cycle of $G$.
    Let $S \subseteq V(G)$, $S \neq \emptyset$, $t :=$ \# components of $G - S$
    There are at least 2 edges of $ C $ between each component of $ G - S $
    and $ S$. If $ e = \#$ edges of $ C $ between $ S $ and $ V - S$, 
    we have 
    $$ e \geq t \cdot 2 \text{ and } e \underset{C is 2-\text{regular}}{\leq}
    |S| \cdot 2 $$ 
\end{proof}

\paragraph{Theorem 10.2 Dirac} Every graph with $n \geq 3$ vertices and 
minimum degree at least $\frac{n}{2}$ has a Hamiltonian cycle.
\begin{proof}
    First we note that $G$ is connected, otherwise a smaller component has all 
    vertices of degree at most $n/2-1$. Consider a longest path 
    $P = (v_0,...,v_k)$. Then $N(v_0),N(v_k) \subseteq V(P)$. Since 
    $|N(_0)|,|N(v_k)| \geq n/2$, and $k \leq n-1$, we have by pigeonhole
    principle that $v_0v_k \in E(G)$ or there is $i, 0 < i < k-1$ such 
    that $v_0v_{i+1} \in E(G)$ and $v_iv_k \in E(G)$. In any case there is a 
    cycle $C$ on $k+1$ vertices in $G$. If $k+1=n$ we are done. If $k+1<n$, 
    since $G$ is connected there is a vertex $v$ not in $C$ that is adjacent to
    a vertex with $C$. Then $v$ and $C$ induce a graph that contains a spanning
    path, i.e. a path on $k+2$ vertices, a contradiction to maximality of $P$.
\end{proof}

\paragraph{Ore's Thm.}
A graph $ G $ on $ n \geq 3 $ vertices is Hamiltonian if  
$ \forall u,v \in V(G), uv \notin E(G), d(u)+d(v) \geq n $
