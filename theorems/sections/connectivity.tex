\section{Connectivity}

\paragraph{Definitions}
\begin{itemize}
    \item for a natural number $ k \geq 1 $, a graph $ G $ is called 
    \textit{k-connected} if $ |V(G)| \geq k + 1 $ and for any set $ U $ of $ k - 1 $
    vertices in $ G $ the graph $ G - U $ is connected. In particular $ K_n $ is
    $(n-1)$-connected.

    \smallskip \noindent
    this implies: a graph is $k$-connected if any two of its vertices 
    can be joined by $k$ independent paths

    \item a graph $ G $ is called $k$-$linked$ if $ |G| \geq 2k $ and for any $ 2k $ 
    distinct vertices $s_1,...,s_k,t_1,...,t_k$ there are vertex disjoint 
    $s_i$-$t_i$-paths, $i = 1,...,k$
    
\end{itemize}

\paragraph{Lemma 3.2} For any connected, non-trivial graph $ G $ we have
$$ \kappa(G) \leq \kappa^\prime(G) \leq \delta(G) $$
\begin{proof}
   If $ G $ is complete $\kappa(G) = \kappa^\prime(G) = \delta(G) = n - 1$ 

   \smallskip
   Assume $ G $ is not complete: \\
   $ \kappa^\prime(G) \leq \delta(G) $: \\
   Simply remove all the edges of a vertex with minimum degree.

   \smallskip
   $ \kappa(G) \leq \kappa^\prime(G)$: \\
   Consider smallest separating set of edges $ F $:

   \smallskip \noindent
   case1: \\
   there is a vertex $ v $ not incident to $ F $, then the vertices incident
   to $ F $ separate this vertex from any vertex in the other component.


   \smallskip \noindent
   case2: \\
   Every vertex is incident to $ F $, consider $ v $ of degree $ < |G| - 1 $, 
   exists because $ G $ is not complete. Show the neighbourhood of $ v $ 
   is less than $ |F| $ so $ N(v) $ is a separating set.
\end{proof}

\paragraph{Definition} A subset $ X $ of vertices and edges of $ G $ seperates
two vertex sets $ A,B $ if each $A$-$B$-path (starts in $ A $ ends in $B$) contains an element of $X$.

\paragraph{Theorem 3.3 (Menger's Theorem)} For any graph $ G $ and any two 
vertex sets $ A,B \subseteq V(G) $, the smallest number of vertices separating 
$ A $ and $ B $ is equal to the largest number of disjoint $A$-$B$-paths.

\paragraph{Theorem 3.4 (Global Version of Menger's Theorem)} A graph $ G $ is 
$k$-connected if and only if for any two vertices $ a,b $ in $ G $ there exist 
$ k $ independent $a$-$b$-paths.

\smallskip \noindent
Note that Menger's Theorem implies that if $ G $ is $k$-linked, then $ G $ 
is $k$-connected.

\paragraph{Definition: Line Graph} For a graph $ G = (V,E) $ the line graph
$L(G)$ is the graph $L(G) = (E,E^\prime) $ where
$$ E^\prime = \{\{e_1,e_2\} \in \binom{E}{2}: e_1 
\text{ adjacent to } e_2 \text{ in } G\} $$

\paragraph{Corollary 24} If $a,b$ are vertices of $ G$, then
\begin{center}
    min \#edges separating $ a $ and $ b $ = max \#edge-disjoint $a$-$b$-paths
\end{center}


\paragraph{Definition}
\begin{itemize}
    \item \textit{H-path}: Given a graph $ H $, we call a path $ P $ an $H-path$ 
    if $ P $ is non-trivial and meets $ H $ exactly in its ends. Such a path is also 
    called an $ ear $ of the graph $ H \cup P$.
    \item An $ear$-$decomposition$ of a graph $ G $ is a sequence $ G_0 \subseteq 
    G_1 \subseteq ... \subseteq G_k $ of graphs, such that 
        \begin{itemize}
            \item $G_0$ is a cycle
            \item for each $ i = 1,...,k$ the graph $ G_i $ arises from $ G_{i-1}$
            by adding a $ G_{i-1}$-path $P_i$, i.e. $P_i$ is an ear of $G_i$
            \item $G_k = G$
        \end{itemize} 
\end{itemize}

\paragraph{Theorem 25 (Ear-decomposition)} A graph $ G $ is 2-connected if and only 
if it has an ear decomposition starting from any cycle of $ G$.
\begin{proof}
    $\Leftarrow$: \\
    Assume first that $G$ has a ear-decomposition starting from a cycle $C$, i.e.
    $C = G_0 \subseteq G_1 \subseteq ... \subseteq G_k = G$ Use induction on $i$
    to show that $G_i$ is 2-connected.
    $G_0$ is a cylce and therefore 2-connected. $G_{i+1}$ is obtained from $G_i$
    by adding an ear. We now by IH that $G_i$ is 2-connected, therefore a cut vertex 
    has to be on the ear. The ear is contained in a cycle therefore deleting a 
    vertex from it does not disconnect $G_{i+1}$ therefore $G_{i+1}$ is 2-connected.

    \bigskip \noindent
    $\Rightarrow$: \\
    Assume $G$ is 2-connected and $C$ is a cycle in $G$. Let $H$ be the largest 
    subgraph of $G$ obtained by ear decomposition starting with $C$. $H$ has to 
    be an induced subgraph of $G$ otherwise an edge incident to two vertice of 
    $H$ would be and ear that could be added to $H$.

    \smallskip \noindent
    Assume $H \neq G$. Since $G$ is connected there is an edge from $v \in H$ to 
    $u \in G \setminus H$. Since $G - u$ is connected consider a 
    $v$-$w$-path $P$ in $G-u$ for some vertex $w \in (V(H) - u)$. Let 
    $w^\prime$ be the first vertex from $V(H)-u$ on this path. Then 
    $Pw^\prime \cup uv$ is an ear of $H$, a contradiction to minimality of $H$.
\end{proof}

\paragraph{Lemma 26} If $ G $ is 3-connected with $ G \neq K_4 $, then there exists 
an edge $ e $ of $ G $ such that $ G \circ e $ is also 3-connected.

\paragraph{Theorem 3.6 (Tutte)} A graph $ G $ is 3-connected if and only if there 
exists a sequence of graphs $ G_0,G_1,...,G_k$, such that 
\begin{itemize}
    \item $G_0 = K_4$
    \item for each $ i = 1,...,k$ the graph $ G_i $ has two adjacent vertices $ x^\prime, 
    x^{\prime\prime} $ of degree at least 3, so that $ G_{i-1} = G_i \circ 
    x^{\prime}x^{\prime\prime} $
    \item $ G_k = G$
\end{itemize}
\begin{proof} $ $ \\
    $\Rightarrow$: \\
    If $G$ is 3-connected, such a sequence exists by Lemma 26. To see the 
    degree condition is satisfied, recall that $\delta(H) \geq 3$ for any 
    3-connected graph $H$. Note that with each contraction, the number of 
    vertices decrease by 1 and until we have at least 5 vertices, we can 
    apply Lemma 26 and contract one more edge. Thus we stop at a graph
    $G_0$ which has 4 vertices and $\delta(G_0) \geq 3$ from which 
    $G_0 \cong K_4$ follows.

    \bigskip
    $\Leftarrow$ \\
    We consider a sequence of graphs satisfying the given conditions and show 
    that each graph in the sequence is 3-connected. Assume that $G_i$ is 
    3-connected, $G_{i+1}$ is not, and $G_i = G_{i+1} \circ xy$, for an edge $xy$
    of $G_{i+1}$ such that $d(x),d(y) \geq 3$. Then $G_{i+1}$ has a cut-set 
    $S$ with at most two vertices.

    \smallskip \noindent
    Case 1: $x,y \in S$. \\
    Then $G_i$ has a cut vertex, a contradiction.

    \smallskip \noindent
    Case 2: $x \in S, y \notin S, y$ is not the only vertex of its component in 
    $G_{i+1} - S$.\\
    Then $G_i$ has a cut set of size at most 2, a contradiction.

    \smallskip \noindent
    Case 3: $x \in S, y \notin S, y$ is the only vertex of its component in 
    $G_{i+1} - S$.\\
    Then $d(y) \leq 2$, a contradiction to the fact that $d(y) \geq 3$

    \smallskip \noindent
    Case 4: $x,y \notin S$\\
    Then $x$ and $y$ are in the same component of $G_{i+1} - S$. So $S$ is 
    a cut set of $G_i$, a contradiction.
\end{proof}
Note that this theorem gives a way to generate all 3-connected graphs by starting with 
$K_4$ and creating a sequence of graphs by \enquote{uncontracting} a vertex such that 
the degrees of new vertices are at least 3 each.


\paragraph{Theorem 27 (Mader)} Every graph $ G = (V,E) $ of average degree at least $4k$
has a $k$-connected subgraph.


\paragraph{Definition}
\begin{itemize}
    \item let $ G $ be a graph, a maximal connected subgraph of $ G $ without a 
    cut vertex is called a $block$ of $ G$. In particular, the blocks of $ G $ 
    are exactly the bridges and the maximal 2-connected subgraphs of $ G$.
    \item \textit{block-cut-vertex graph or block graph} of $ G $ is a bipartite
    graph $ H $ whose partite sets are the $blocks$ of $ G $ and the cute vertices
    of $ G $ respectively. There is an edge between a block $ B $ and a cut vertex 
    $ a $ if and only if $ a \in B $, i.e. the block contains the cut vertex. The 
    leaves of this graph are called $leaf$ $blocks$.
\end{itemize}

\paragraph{Theorem 28} The block-cut-vertex graph of a connected graph is a tree.