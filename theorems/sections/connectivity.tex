\section{Connectivity}

\paragraph{Definitions}
\begin{itemize}
    \item for a natural number $ k \geq 1 $, a graph $ G $ is called 
    \textit{k-connected} if $ |V(G)| \geq k + 1 $ and for any set $ U $ of $ k - 1 $
    vertices in $ G $ the graph $ G - U $ is connected. In particular $ K_n $ is
    $(n-1)$-connected.

    \smallskip \noindent
    this implies: a graph is $k$-connected if any two of its vertices 
    can be joined by $k$ independent paths

    \item a graph $ G $ is called $k$-$linked$ if $ |G| \geq 2k $ and for any $ 2k $ 
    distinct vertices $s_1,...,s_k,t_1,...,t_k$ there are vertex disjoint 
    $s_i$-$t_i$-paths, $i = 1,...,k$
    
\end{itemize}

\paragraph{Lemma 3.2} For any connected, non-trivial graph $ G $ we have
$$ \kappa(G) \leq \kappa^\prime(G) \leq \delta(G) $$
\begin{proof}
   If $ G $ is complete $\kappa(G) = \kappa^\prime(G) = \delta(G) = n - 1$ 

   \smallskip
   Assume $ G $ is not complete: \\
   $ \kappa^\prime(G) \leq \delta(G) $: \\
   Simply remove all the edges of a vertex with minimum degree.

   \smallskip
   $ \kappa(G) \leq \kappa^\prime(G)$: \\
   Consider smallest separating set of edges $ F $:

   \smallskip \noindent
   case1: \\
   there is a vertex $ v $ not incident to $ F $, then the vertices incident
   to $ F $ separate this vertex from any vertex in the other component.


   \smallskip \noindent
   case2: \\
   Every vertex is incident to $ F $, consider $ v $ of degree $ < |G| - 1 $, 
   exists because $ G $ is not complete. Show the neighbourhood of $ v $ 
   is less than $ |F| $ so $ N(v) $ is a separating set.
\end{proof}

\paragraph{Definition} A subset $ X $ of vertices and edges of $ G $ seperates
two vertex sets $ A,B $ if each $A$-$B$-path (starts in $ A $ ends in $B$) contains an element of $X$.

\paragraph{Theorem 3.3 (Menger's Theorem)} For any graph $ G $ and any two 
vertex sets $ A,B \subseteq V(G) $, the smallest number of vertices separating 
$ A $ and $ B $ is equal to the largest number of disjoint $A$-$B$-paths.

\paragraph{Theorem 3.4 (Global Version of Menger's Theorem)} A graph $ G $ is 
$k$-connected if and only if for any two vertices $ a,b $ in $ G $ there exist 
$ k $ independent $a$-$b$-paths.

\smallskip \noindent
Note that Menger's Theorem implies that if $ G $ is $k$-linked, then $ G $ 
is $k$-connected.

\paragraph{Definition: Line Graph} For a graph $ G = (V,E) $ the line graph
$L(G)$ is the graph $L(G) = (E,E^\prime) $ where
$$ E^\prime = \{\{e_1,e_2\} \in \binom{E}{2}: e_1 
\text{ adjacent to } e_2 \text{ in } G\} $$

\paragraph{Corollary 24} If $a,b$ are vertices of $ G$, then
\begin{center}
    min \#edges separating $ a $ and $ b $ = max \#edge-disjoint $a$-$b$-paths
\end{center}


\paragraph{Definition}
\begin{itemize}
    \item \textit{H-path}: Given a graph $ H $, we call a path $ P $ an $H-path$ 
    if $ P $ is non-trivial and meets $ H $ exactly in its ends. Such a path is also 
    called an $ ear $ of the graph $ H \cup P$.
    \item An $ear$-$decomposition$ of a graph $ G $ is a sequence $ G_0 \subseteq 
    G_1 \subseteq ... \subseteq G_k $ of graphs, such that 
        \begin{itemize}
            \item $G_0$ is a cycle
            \item for each $ i = 1,...,k$ the graph $ G_i $ arises from $ G_{i-1}$
            by adding a $ G_{i-1}$-path $P_i$, i.e. $P_i$ is an ear of $G_i$
            \item $G_k = G$
        \end{itemize} 
\end{itemize}

\paragraph{Theorem 25 (Ear-decomposition)} A graph $ G $ is 2-connected if and only 
if it has an ear decomposition starting from any cycle of $ G$.
\begin{proof}
{\color{red}{TODO}}
\end{proof}

\paragraph{Lemma 26} If $ G $ is 3-connected with $ G \neq K_4 $, then there exists 
an edge $ e $ of $ G $ such that $ G \circ e $ is also 3-connected.
\begin{proof}
 {\color{red}{TODO}}
\end{proof}

\paragraph{Theorem 3.6 (Tutte)} A graph $ G $ is 3-connected if and only if there 
exists a sequence of graphs $ G_0,G_1,...,G_k$, such that 
\begin{itemize}
    \item $G_0 = K_4$
    \item for each $ i = 1,...,k$ the graph $ G_i $ has two adjacent vertices $ x^\prime, 
    x^{\prime\prime} $ of degree at least 3, so that $ G_{i-1} = G_i \circ 
    x^{prime}x^{\prime\prime} $
    \item $ G_k = G$
\end{itemize}
\begin{proof}
  {\color{red}{TODO}}
\end{proof}

\paragraph{Theorem 27 (Mader)} Every graph $ G = (V,E) $ of average degree at least $ 4k $
has a $k$-connected subgraph.
\begin{proof}
    
\end{proof}

\paragraph{Definition}
\begin{itemize}
    \item let $ G $ be a graph, a maximal connected subgraph of $ G $ without a 
    cut vertex is called a $block$ of $ G$. In particular, the blocks of $ G $ 
    are exactly the bridges and the maximal 2-connected subgraphs of $ G$.
    \item \textit{block-cut-vertex graph or block graph} of $ G $ is a bipartite
    graph $ H $ whose partite sets are the $blocks$ of $ G $ and the cute vertices
    of $ G $ respectively. There is an edge between a block $ B $ and a cut vertex 
    $ a $ if and only if $ a \in B $, i.e. the block contains the cut vertex. The 
    leaves of this graph are called $leaf$ $blocks$.
\end{itemize}

\paragraph{Theorem 28} The block-cut-vertex graph of a connected graph is a tree.