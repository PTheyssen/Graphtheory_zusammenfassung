\section{Basic Probability}

\paragraph{Definitions}
\begin{itemize}
    \item \textit{finite probability space} consists of a finite
    set $ \Omega $ and a function (called \textit{probability distribution})
    Pr$: \Omega \to [0,1] $, such that $ \sum_{x \in \Omega} \text{Pr}[x] = 1 $
    \item we call $\Omega$ the $domain$ (or a $sample$ $space$).
\end{itemize}
A probability space is a representation of a random experiment, where
we choose a member of $ \Omega $ at random and Pr$[x]$ is the 
probability that $ x $ is chosen. Subsets $ A \subseteq \Omega $ 
are called $events$. The probability of an event is defined by 
Pr$[A] := \sum_{x\in A} \text{Pr}[x]$, i.e. the probability that 
a member of $ A $ is chosen. 

\smallskip
The most common probability distribution is the $uniform$ $distribution$,
which is defined as Pr$[x] = 1/|\Omega|$ for each $ x \in \Omega$; the 
corresponding sample space is then called $symmetric$.

\paragraph{Elementary properties} For any two events $ A $ and $ B $ 
we have that 
\begin{enumerate}
    \item Pr$[\Omega] = 1$, Pr$[\emptyset] = 0$, and Pr$[A] \geq 0$ 
    for all $A \subseteq \Omega$
    \item Pr$[A \cup B] =$ Pr$[A] \;+ $ Pr$[B] \;-$ Pr$[A \cap B] \leq
    $ Pr$[A] \;+ $ Pr$[B]$
    \item Pr$[A \cup B] =$ Pr$[A] \;+ $ Pr$[B]$ if $ A $ and $ B $ 
    are disjoint 
    \item Pr$[\overline{A}] = 1 \; - $ Pr$[A]$
    \item Pr$[A \setminus B] = $ Pr$[A] \; -$ Pr$[A \cap B]$
    \item Pr$[A \cap B] \geq $ Pr$[A] \; - $ Pr$[\overline{B}]$
    \item If $ B_1,...,B_m $ is a partition of $ \Omega $ then 
    Pr$[A] = \sum_{i = 1}^m$ Pr$[A \cap B_i] $
\end{enumerate}

\paragraph{De Morgan Rule}
$$ (A \cup B)^c = A^c \cap B^c $$ 
$$ (A_1 \cup A_2 \cup ... \cup A_n)^c = A_1^c \cap A_2^c \cap ... \cap A_n^c $$

\paragraph{Conditional Probability} $ $ \\
For two events $ A $ and $ B $, the $conditional$ $probability$ of $ A $ 
given $ B $, denoted Pr$[A|B]$, is the probability that one would assign
to $ A $ if one knew that $ B $ occurs.
$$ \text{Pr}[A|B] := \frac{\text{Pr}[A \cap B]}{\text{Pr}[B]} $$
$$ \frac{\text{Amount of tries, where A and B happen}}{\text{Amount of tries where B happens}} $$

\bigskip \noindent
Two events $ A $ and $ B $ are called $independent$ if Pr$[A \cap B] 
= \text{Pr}[A] \cdot \text{Pr}[B]$ if $ B \neq \emptyset $ this is 
equivalent to $ \text{Pr}[A|B] = \text{Pr}[A]$.

\paragraph{Random Subset}
A $random$ $subset$ $ S $ of $ \Gamma $ is obtained by flipping a coin, 
with probability $ p $ of success, for each element of $ \Gamma $ 
to determine whether the element is to be included in $ S $; the 
distribution of $ S $ is the probability distribution on $ \Omega = 2^\Gamma $
given Pr$[S] = p^{|S|}(1-p)^{|\Gamma| - |S|} $ for $ S \subseteq \Gamma $.
We will mainly consider the case when $ S $ is uniformly distributed, that is 
when $ p = 1/2$. In this case each subset $ S \subseteq \Gamma $ receives 
the same probability Pr$[S] = 2^{-|\Gamma|} $ 

\smallskip 
A \textit{random variable} is a variable defined as a function 
$ X: \Omega \to \mathbb{R} $ of the domain of a probability space.