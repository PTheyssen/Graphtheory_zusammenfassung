\section{Random Graphs}
\paragraph{Definitions}
\begin{itemize}
    \item Erdos-Renyi model of random graphs $\mathcal{G}(n,p)$ is the 
    probability space on all $n$-vertex graphs that results from 
    independently deciding whether to include each of the $\binom{n}{2}$
    possible edges with fixed probability $p \in [0,1]$    

    \item A property $\mathcal{P}$ is a set of graphs, for example 
    $\mathcal{P} = \{G: G \text{ is } k\text{-connected}\}$

    \item Let $(p_n) \in [0,1]^{\mathbb{N}}$ be a sequence.
    We say that $G \in \mathcal{G}(n,p_n)$ $almost$ $always$ has a property
    $\mathcal{P}$ if Prob$(G \in \mathcal{G}(n,p_n) \cap \mathcal{P}) \to 1$
    for $n \to \infty$. If $(p_n)$ is constant $p$, we also say in this case
    that $almost$ $all$ graphs in $\mathcal{G}(n,p)$ have property
    $\mathcal{P}$

    \item A function $f(n): \mathbb{N} \to [0,1]$ is  a 
    \textit{threshold function} for property $\mathcal{P}$ if:
        \begin{itemize}
            \item for all $(p_n) \in [0,1]^{\mathbb{N}}$ with 
            $$\frac{p_n}{f(n)} \underset{n\to\infty}{\rightarrow} 0$$ the graph 
            $G \in \mathcal{G}(n,p_n)$ almost always does not have property 
            $\mathcal{P}$

            \item for all $(p_n) \in [0,1]^{\mathbb{N}}$ with 
            $$\frac{p_n}{f(n)} \underset{n\to\infty}{\rightarrow} \infty$$ the graph 
            $G \in \mathcal{G}(n,p_n)$ almost always has property 
            $\mathcal{P}$

            \item threshold function for containing a cycle is $f(n) = 1/n$
            meaning if $p > 1/n$ we almost always have a cycle

        \end{itemize}
\end{itemize}

\paragraph{Usage of probabilistic Method}
\begin{itemize}
    \item ex$(n,K_{t,t})$
    \item $\sqrt{2}^k \leq R(k)$
    \item Erd\H{o}s-Hajnal
\end{itemize}

\paragraph{Lemma 107} Ed\H{o}s-Renyi model is universal
{\color{red}{TODO}}
Lemma 107 tells us that Erdos-Renyi model is universal, it gives us any graph.

\paragraph{Lemma 108}
{\color{red}{TODO}}

\paragraph{Lemma 110} Expected number of cycles \\
{\color{red}{TODO}}

\paragraph{Theorem 9.2 (Erd\H{o}s)} 
For any $k \geq 2$ there is a Graph $G$ on $\sqrt{2}^k$ vertices such that 
$\alpha(G) < k$ and $\omega(G) < k$. This implies $R(k,k) \geq 2^{k/2}$.
\begin{proof}
    {\color{red}{TODO}}   
\end{proof}


\paragraph{Theorem 9.3 (Erdos-Hajnal)} 
sketch: take random graph, number of short cycles (length at most k) is < $ \frac{n}{2} $ 
delete a vertex from each of these cycles and get a graph $G^\prime $ so the girth
of $ G^\prime $ > $ k $.
We know chromatic number of G prime is $ \geq \frac{|V(G^\prime)|}{\alpha(G^\prime)} $
with lemma 108 we can show $ \alpha $ is not so big and therefore get a lower bound

\smallskip \noindent
\textbf{two tricks:} 
\begin{itemize}
    \item random graph and kill short cycle 
    \item bound the independence number (co-clique number) size of the largest 
    independent set
\end{itemize}