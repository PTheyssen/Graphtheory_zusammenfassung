\section{Random Graphs}
Example: $ n = 4 $ and $ p = \frac{1}{3} $
\\
empty graph has prob $ = \frac{2}{3}^6 $

\smallskip
graph with 2 edges has prob $ \frac{1}{3}^2 \cdot \frac{2}{3}^4 $

\paragraph{Note}
let m be number of edges in G
$$ \sum_{G graph on [n]} Prob(G) = \sum_{m=0}^{\binom{n}{2}} \binom{\binom{n}{2}}{m}
\cdot p^m (1-p)^{\binom{n}{2}-m} = (p + (1 -p)^{\binom{n}{2}}) = 1 $$

\smallskip
Lemma 107 tells us that Erdos-Renyi model is universal, it gives us any graph.

\smallskip
The type of estimats of lemma 108 are used in many different results.

\smallskip
Lemma 109: count cycle of $ k $ vertices overcounting for k in also in different order 
therefore divide by $ 2k $ and times $ p^k $ because we have $ k $ edges in our cycle

\smallskip
purpose: for this lemma, to prove ramsey result, but the main purpose is the result
of erdos-Hajnal (thm 9.3) that graph with arbitrarily high girth (no cycle of length smaller ) and abitrarily high chromatic number.


\paragraph{Theorem 9.3 (Erdos-Hajnal)} 
sketch: take random graph, number of short cycles (length at most k) is < $ \frac{n}{2} $ 
delete a vertex from each of these cycles and get a graph $G^\prime $ so the girth
of $ G^\prime $ > $ k $.
We know chromatic number of G prime is $ \geq \frac{|V(G^\prime)|}{\alpha(G^\prime)} $
with lemma 108 we can show $ \alpha $ is not so big and therefore get a lower bound

\smallskip
two tricks: 
\begin{itemize}
    \item random graph and kill short cycle 
    \item bound the independence number (co-clique number) size of the largest 
    independent set
\end{itemize}

\bigskip 
papers on probablistic method are not easy to read, because it involves choosing the 
parameters correctly and asymptotic analysis. But the results obtained by 
probablistic mehtod cannot be obtained by any other method.

\paragraph{Graph Properties and threshold functions}
\textit{graphproperty} is a set of graphs. 
 Example: $$\textit{P} = \{G: \text{ G is connected}\}  $$

 
