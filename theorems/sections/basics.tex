\section{Basics}

\paragraph{Definitions}
\begin{itemize}
    \item A graph $ G $ is \textit{non-trivial} if it contains at least one edge,
    equivalently if $ G $ is not an empty graph 
    \item The \textit{order} of $ G $ writen $ |G| $, is the number of vertices of 
    $ G $, i.e. $ |G| = |V| $
    \item The \textit{size} of $ G $ wirten $ ||G|| $, is the number of edges of 
    $ G $, i. e. $ ||G|| = |E| $, if order of $ G $ is $ n $ then the size of $ G $ 
    is between 0 and $ \binom{n}{2} $
    \item $ N(S) $ the \textit{neighbourhood} of $ S \subseteq V $ is the set of
    vertices in $ V $. that have and adjacent vertex in $ S $. 
    Instead of $ N(\{v\}) $ for $ v \in V $ we write $ N(v) $
    \item vertex of degree 1 is called \textit{leaf}
    \item vertex of degree 0 is called \textit{isolated vertex}
    \item \textit{minimum degree of G}, denoted by $ \delta(G) $ is the smallest
    vertex degree in $ G $
    \item \textit{maximum degree of G}, denoted by $ \Delta(G) $ is the highest
    vertex degree in $ G $
    \item graph $ G $ is called \textit{k-regular}, with $ k \in \mathbb{N} $, 
    if all vertices have degree $ k $.
    \item \textit{average degree of G} is defined as $ d(G) = 
    \frac{\sum_{v \in V} deg(v)}{|V|} $ \\ We have 
    $$ \delta(G) \leq d(G) \leq \Delta(G) $$  
    with equality if and only if $ G $ is $ k$-regular

\end{itemize}

\paragraph{Handshake Lemma}
For ever graph $ G = (V,E) $ we have
$$ 2|E| = \sum_{v\in V} d(v) $$ 
\begin{proof}
   By double counting the set $ X = \{(e,x) : e \in E(G), x \in V(G), x \in e\}  $ then
   $$ |X| = \sum_{v\in V(G)} d(x) $$ 
   and 
   $$ |X| = \sum_{e\in E(G)} 2 = 2|E(G)|$$
   by the principle of double counting the terms are equal.
\end{proof}

\paragraph{Corollary} From this follows that the sum of all vertex degrees is even and 
therefore the number of vertices with odd degree is even.

\paragraph{Proposition 3} If a graph G has minimum degree $ \delta(G) \geq 2 $, then
$ G $ has a path of length $ \delta(G) $ and a cycle with at least $ \delta(G) + 1 $
vertices.
\begin{proof}
Let $ P = (x_0,...,x_k) $ be a longest path in $ G $. Then $ N(x_0) \subseteq V(P)$,
otherwise for $ x \in N(x_0) \setminus V(P) $ the path $ (x,x_0,x_1,...,x_k) $ would
be a longer path.

\smallskip
Let $ i $ be the largest index such that $ x_i \in N(x_0) $, then $ i \geq |N(x_0)| 
\geq \delta $. So $ (x_0,x_1,...,x_i,x_0) $ is a cycle of length at least 
$ \delta(G) + 1$.
\end{proof}

\paragraph{Proposition 4} If for distinct vertices $ u $ and $ v $ a graph has 
a $u$-$v$-walk, then it has a $u$-$v$-path.
\begin{proof}
    Consider a $u$-$v$-walk $ W $ with the smallest number of edges. Assume that $ W $
    does not form a path, then there is a repeated vertex, $ w $, i.e.
    $$ W = u,e,v_1,e_1,...,e_k,w,e_{k+1},...,e_l,w,e_{l+1},...,v $$
    Then $ W_1 = u,e,v_1,...,e_k,w,e_{k+1},...,v $ is a shorter $u$-$v$-walk, 
    a contradiction.
\end{proof}


\paragraph{Propostion 5} If a graph has a closed walk of odd length, then it contains 
an odd cycle.
\begin{proof}
   Let $ W $ be the shortest closed odd walk. If $ W $ is a cycle the Proposition holds. 
   Otherwise there is a repeated vertex, so $ W $ is an edge-disjoint union of 
   two closed walks. The sum of the lengths of these walks is odd, therefore one 
   of them is an odd closed walk shorter than $ W $ a contradiction to the minimality
   of $ W $.
\end{proof}

\paragraph{Proposition 6} If a graph has a closed walk with a non-repeated edge, then
the graph contains a cycle.
\begin{proof}
    Let $ W $ be a shortest walk with a non-repeated edge $ e $. If $ W $ is a cycle, 
    we are done. Otherwise, there is a repeated vertex and $ W $ is a union of 
    two closed walks $ W_1 $ and $ W_2 $ that are shorter than $ W $. One of them
    say $ W_1 $, contains $ e$, a non-repeated edge. This contradicts the 
    minimality of $ W$.
\end{proof}

\paragraph{Definition bipartite}
A graph $ G = (V,E) $ is called $ bipartite $ if there exists natural numbers $ m,n $
such that $ G $ is isomorphic to a subgraph of $ K_{m,n} $. Then the vertex set 
can be written as $ V = A \cupdot B $ such that $ E \subseteq \{ab: a\in A,
b \in B\}$. The sets $ A $ and $ B $ are called the $ partite $ $ sets $ of $ G $



\paragraph{Proposition 1.5} A graph is bipartite if and only if it has no cycles 
of odd length.
\begin{proof}{skript} \\
    \enquote{$\Rightarrow$}  \\
    Assume that $ G $ is a bipartite graph with parts $ A $ and $ B $. Then any cycle
    has a form $a_1,b_1,a_2,b_2,...,a_k,b_k,a_1 $ where $ a_i \in A, b_i \in B, 
    i \in [k]$. Thus every cycle has even length. 

    \smallskip
    \noindent\enquote{$\Leftarrow$} \\
    Assume $ G $ does not have cycles of odd length. We can assume that $ G $ is 
    connected, otherwise we can treat the connected components separately.
    Let $ v \in V(G) $.
    Let $ A = \{u \in V(G) : dist(u,v) \equiv 0 \text{ (mod } 2)\} $ and let
    $ B = \{u \in V(G) : dist(u,v) \equiv 1 \text{ (mod } 2)\} $  We claim that $ G $ is 
    bipartite with parts $ A $ and $ B $. To verify this it is sufficient to 
    prove that $ A $ and $ B $ are independent sets.
    Let $u_1u_2 \in E(G) $ and let $ P_1 $ be a shortest $u_1$-$v$-path and 
    $ P_2 $ a shortest $u_2$-$v$-path. Then the union of $ P_1, P_2 $ 
    and $ u_1u_2 $ forms a closed walk $ W $. If $ u_1, u_2 \in A $ or 
    $ u_1, u_2 \in B $ then $ W $ is a closed odd walk, because $ dist(v, u_1) $ 
    and $ dist(v, u_2) $ are both even or odd. Thus by Prop. 5 $ G $ contains
    an odd cycle, a contradiction. Thus for any edge $ u_1u_2 $ the adjacent
    vertices $ u_1 $ and $ u_2 $ are in different parts $ A $ or $ B $.
    Therefore $ A $ and $ B $ are independent sets. 
\end{proof}
\begin{proof}{Diestel} $ $  \\
    \enquote{$\Leftarrow$} \\
    Let $ T $ be a spanning tree in $ G $, pick a root $ r \in T $ and denote the 
    associated tree-order on $ V $ by $ \leq_T $ (this order expressing height if 
    $ x < y $ then $ x $ lies $ below $ $ y $ in $ T $). For each $ v \in V(G) $ the 
    unique path $r$-$v$-$T$ has odd or even length.
    This defines a bipartition of $ V(G) $, we show that $ G $ is bipartite with this
    partition.
    Let $ e = xy $ be an edge of $ G$. If $ e \in T $ with $ x <_T y $ say, then 
    $r$-$y$-$T$ = $r$-$xy$-$T$ and so $ x $ and $ y $ lie in different partition 
    classes. If $ e \notin T $ then $ C_e := x$-$y$-$T +$ $e $ is a cycle, and
    by the case treated already the vertices along $x$-$y$-$T$ alternate between 
    the two classes. Since $ C_e $ is even by assumption, $ x $ and $ y $ again lie
    in different classes. 
\end{proof}

\paragraph{Euler tour} A closed walk that traverses every edge of the graph 
exactly once is called an $ Euler \text{ }tour $.
\paragraph{Theorem 1.6 (Eulerian Tour Condition)} A connected graph has an Eulerian 
Tour if and only if every vertex has even degree.
\begin{proof}
   \enquote{$\Rightarrow$} \\ 
   The degree condition is necessary for an euler tour, because a vertex appearing
   $ k $ times in an Euler tour (or $ k + 1 $ times if it is the starting and 
   finishing vertex) must have degree $ 2k $.

   \smallskip
   \noindent\enquote{$\Leftarrow$} \\
   Show by induction on $ ||G|| $ that every connected Graph $ G $ with all degrees
   even has an Euler tour. $ ||G|| = 0 $ is trivial. 

   \smallskip
   Now let $ ||G|| \geq 1 $, since all degrees are even, we can find in $ G $
   a non-trivial closed walk that contains no edge more than once. To find this 
   walk we consider $ W $ a walk of maximal length and write $ F $ for 
   the set of its edges. If $ F = E(G) $, then $ W $ is an Euler tour.
   
   \smallskip
   Suppose, therefore $ G^\prime := G - F $ has an edge. \\
   For every vertex $ v \in G $, an even number of edges of $ G $ at $ v $ lies 
   in $ F $, so the degrees of $ G^\prime $ are again all even.
   Since $ G $ is connected, $ G^\prime $ has an edge $ e $ incident with a vertex on
   $ W $. By I.H. the component $ C $ of $ G^\prime $ containing $ e $ has an Euler
   tour. Concatenating this with $ W $ (suitably re-indexed), we obtain a closed walk
   in $ G $ that contradicts the maximal length of $ W.$

   \end{proof}


\paragraph{Definitions}
\begin{itemize}
    \item graph $ G $ is $ connected $ if any two vertices are linked by a path. 
    \item a maximal connected subgraph of $ G $ is called a $ connected $ 
    $ component $ of $ G$.
    \item acyclic graphs are called $ forests $
    \item a graph $ G $ is called a $ tree $ if $ G $ is connected and acyclic.
\end{itemize}

\paragraph{Lemma 7} Every tree on at least two vertices has a leaf.
\begin{proof}
    If a tree $ T $ on at least two vertices does not have leaves
    then every vertex has degree > than 2, so we have a cycle in $ T $ 
    with length $ \geq $ 3, a contradiction.
\end{proof}

\paragraph{Lemma 8} A tree of order $ n \geq 1 $ has exactly $ n - 1 $ edges.
\begin{proof}
    We prove the statement by induction on $ n$. When $ n = 1$, there are no edges.

    \smallskip
    \noindent\textbf{I.H.:} Assume that each tree on $ n = k $ vertices has 
    $ k - 1 $ edges, with $ k \geq 1 $.
    
    \smallskip
    \noindent\textbf{Step:} Lets prove that each tree on $ k+1 $ vertices has $ k $ edges.
    Consider a tree $ T $ on $ k + 1 $ vertices. Since $ k + 1 \geq 2 $, $ T $ has 
    a leaf $ v $. Let $ T^\prime = T - \{v\}$. We see that $ T^\prime $ is connected
    because any $u$-$w$-path in $ T $, for $ u \neq v $ and $ w \neq v $, does not 
    contain $ v $. We see also that $ T^\prime $ is acyclic, because deleting 
    vertices from an acyclic graph does not create new cycles. Thus $ T^\prime $ 
    is a tree on $ k $ vertices. By I.H. $ |E(T^\prime)| = k-1$. Thus 
    $ |E(T)| = |E(T^\prime)| + 1 = (k-1)+1 = k.$

\end{proof}

\paragraph{Lemma 9} Every connected graph contains a spanning tree.
\begin{proof}
    Let $ G $ be a connected graph. Consider $ T $, an acyclic spanning subgraph of 
    $ G $ with largest number of edges. If it is a tree we are done.

    \smallskip
    Otherwise, $ T $ has more than one component. Consider vertices $ u $ and 
    $ v $ from different components of $ G$. Consider a shortest $u$-$v$-path $ P $
    in $ G$. Then $ P $ has an edge $ e = xy $ with exactly one vertex $ x $ in 
    one of the components of $ T$. Then $ P $ has an edge $ e = xy $ with exactly
    one vertex $ x $ in one of the components of $ T$. Then $ T \cup \{e\} $ is 
    acyclic. If there would be a cycle, it would contain $ e $, however $ e $ connects
    to components, therefore cannot be part of a cylce ($ e $ would be a repeated 
    edge). Thus $ T \cup \{e\} $ is a bigger spanning acyclic subgraph of $ G $
    contradicting the maximality of $ T $.
\end{proof}

\paragraph{Lemma 10} A connected graph on $ n \geq 1 $ vertices and $ n-1 $ edges is 
tree.
\begin{proof}
    Let $ G $ be a connected graph on $ n $ vertices with $ n-1 $ edges. 
    Assume $ G $ is not a  tree, i.e. contains a cycle. We therefore 
    can remove a edge so that $ G $ is still connected.
    This is a contradiction because a graph on $ n $ vertices with $ n-2 $ edges 
    cannot be connected. Because a walk from vertex $ 1 $ to vertex $ n $ has to 
    have at least $ n-1 $ edges.
\end{proof}

\paragraph{Lemma 11} The vertices of every connected graph on $ n \geq 2 $ vertices 
can be ordered $ (v_1,...,v_n) $ so that for every $ i \in \{1,...,n\} $ the 
Graph $ G[\{v_1,...,v_i\}] $ is connected.
\begin{proof}{skript} $ $ \\
    Let $ G $ be a connected graph on $ n $ vertices. It contains a spanning tree $ T $.
    Let $ v_n $ be a leaf of $ T $, let $ v_{n-1} $ be a leaf of $ T - \{v_n\} $,
    let $ v_{n-2} $ be a leaf of $ T - \{v_n,v_{n-1}\} $ and so on.
    Let $ v_k $ be a leaf in $ T - \{v_n, v_{n-1},...,v_{k+1}\} $, $ k = 2,...,n $.
    Since deleting a leaf does not disconnect a tree, all resulting graphs form
    a spanning trees of $ G[{v_1,...,v_i}] $, $ i = 1,...,n $. A graph $ H $ having 
    a spanning tree or any connected spanning subgraph $ H^\prime $ is connected
    because a $u$-$v$-path in $ H^\prime $ is a $u$-$v$-path in $ H $. This observation
    completes the proof.
\end{proof}
\begin{proof}{diestel} $ $ \\
    Pick any vertex as $ v_1 $, and assume inductively that $ v_1,...,v_i $ have been 
    chosen for some $ i < |G| $. Now pick a vertex $ v \in G - G_i $. As $ G $ is 
    connected, it contains a $v$-$v_1$ path $ P$. Choose $ v_{i+1} $ as the last vertex
    of $ P $ in $ G - G_i $, then $ v_{i+1} $ has a neighbour in $ G_i $. 
    If we consider $ i+1 $ then we simply add $ v_{i+1} $ to our $ G_i $, Thus
    $ G_{i+1} := G_i \cup \{v_{i+1}\} $ which is also connected.
\end{proof}

\paragraph{Tree equivalences} For any graph $ G = (V,E) $ the following are equivalent:
\begin{enumerate}
    \item $G $ is a tree, i.e. $ G $ is connected and acyclic.
    \item $G $ is connected, but for any $ e \in E $ the graph $ G - e $ is not 
    connected (minimally connected)
    \item $G $ is acyclic, but for any $ x,y \in V(G), xy \notin E(G) $ the graph 
    $ G + xy $ has a cycle. (maximaly acyclic) 
    \item $G $ is connected and 1-degenerate
    \item $G $ is connected and $ |E| = |V| - 1 $
    \item $G $ is acyclic and $ |E| = |V| - 1 $
    \item $G $ is connected and every non-trivial subgraph of $ G $ has a vertex of 
    degree at most 1.
    \item Any two vertices are joined by a unique path in $ G $.
\end{enumerate}
\begin{proof} $ $ \\
    (i) $ \Rightarrow $ (ii): \\
    $ G $ is connected and acyclic, now assume for any edge $ e = xy $ the graph
    $ G^\prime = G - e $ would still be connected. Then $ G^\prime $ has a 
    $x$-$y$-path $ P $. But $ P \cup {e} $ is a cycle in $ G $ which contradicts
    that $ G $ is acyclic.

    \smallskip
    \noindent(ii) $ \Rightarrow $ (i): \\
    $ G $ is connected and for any edge $ e $ the graph $ G - e $ is not connected.
    We want to show that $ G $ is acyclic. If $ G $ would have a cycle we could simply
    remove an edge from that and the resulting graph would still be connected, 
    a contradiction.
\end{proof}
\begin{proof}
    (i) $ \Rightarrow $ (iv): \\


    \smallskip
    \noindent(vi) $ \Rightarrow $ (i): \\
\end{proof}
\begin{proof}
    (i) $ \Rightarrow $ (vii): \\

    \smallskip
    \noindent(vii) $ \Rightarrow $ (i): \\
\end{proof}
\begin{proof}
    (i) $ \Rightarrow $ (viii): \\

    \smallskip
    \noindent(viii) $ \Rightarrow $ (i): \\
\end{proof}


\paragraph{Definition $d$-degenerate}
If there is a vertex ordering $ v_1,...,v_n $ of $ G $ and a $ d \in \mathbb{N} $
such that 
$$ |N(v_i) \cap \{v_{i+1},...,v_n\}| \leq d $$ 
for all $ i \in [n-1] $ then $ G $ is called $ d$-$degenerate $. The minimum
$ d $ for which $ G $ is $ d$-$degenerate $ is called the $ degeneracy $ of $ G $.

\smallskip
\noindent Every finite planar graph has a vertex of degree five or less, therefore 
every planar graph is 5-degenerate.

\paragraph{Definition arboricity}
The least number of trees that can cover the edges of a graph is its arborictiy.

\smallskip
\noindent It is a measure for the graphs maximum local density: it is small if and only if the 
graph is nowhere dense, in the sense that there is no subgraph $ H $ with large 
$ \epsilon(H) = \frac{E(H)}{V(H)} $.

\paragraph{Definition Contract}
For an edge $ e = xy $ in $ G $ we define $ G \circ e $ as the graph obtained from 
$ G $ by identifying $ x $ and $ y $ and removing (if necessary) loops and 
multiple edges. We say that $ G \circ e $ arises from $ G $ by 
\textit{contracting the edge e}.

\paragraph{Definition Complement}
The $ complement $ of $ G $, denoted by $ \overline{G} $ is defined as the graph
$ (V,\binom{V}{2} \setminus E)$. In particular $ G + \overline{G} $ is a 
complete graph and $ \overline{G} = (G + \overline{G}) - E $. 

\paragraph{Definitions} 
\begin{itemize}
    \item \textit{girth of G}, denoted by $ g(G) $ is the length of the shortest cycle
    in $ G $, if $ G $ is acyclic, its girth is said to be $ \infty $
    \item \textit{circumference of G}, is the length of the longest cycle
    if $ G $ is acyclic the circumference is said to be $0$
    \item $ G $ is called $ Hamiltonian $ if $ G $ has a spanning cycle, i.e. 
    a cycle that contains every vertex of $ G $. In other words the circumference is 
    $ |V|$
    \item $ G $ is called $ traceable $ if $ G $ has a spanning path
    \item For two vertices $ v $ and $ u $ in $ G $, the 
    \textit{distance between u and v }, denoted by $ d(v,u) $ is the length of 
    a shortest $u$-$v$-path in $ G $. If no such path exists $ d(u,v) = \infty $
    \item The $ diameter $ of $ G $, denoted by diam($G$), is the maximum distance 
    among all pairs of vertices in $ G $, i.e.
    $$ \text{diam}(G) = \underset{u,v \in V}{\max} \; d(u,v)$$
    \item \textit{eccentricity}, ecc($v$) is the greatest distance of $ v $ to any 
    other vertex.
    \item The \textit{radius of G }, denoted by rad($G$) is defined as 
    $$ \text{rad}(G) =  \underset{u \in V}{\min} \: \underset{n \in V}{\max} \;d(u,v) $$
    its the vertex that has the smallest eccentricity 

\end{itemize}


\paragraph{problem sheets 1 and 2}

\paragraph{problem 1}

\paragraph{problem 2}

\paragraph{problem 3}

\paragraph{problem 4}

\paragraph{problem 5}

\paragraph{problem 6}

\paragraph{problem 7}